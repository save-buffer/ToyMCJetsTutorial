%-------------------------------------------------------------------------------
% This note lists the symbols defined in atlasphysics.sty.
% This variant uses the hepparticle and hepprocess options.
%-------------------------------------------------------------------------------
% Specify where ATLAS LaTeX style files can be found.
\newcommand*{\ATLASLATEXPATH}{../../latex/}
%-------------------------------------------------------------------------------
\documentclass[mhchem,UKenglish,texlive=2016]{\ATLASLATEXPATH atlasdoc}

% Standard packages that can and should be used in ATLAS papers
\usepackage[default, backend=biber]{\ATLASLATEXPATH atlaspackage}
\usepackage{\ATLASLATEXPATH atlasbiblatex}
\usepackage{xtab}
\usepackage{authblk}

% Include all style files
\usepackage[BSM, hion, jetetmiss,
  journal, math, misc, other, hepparticle=true, hepitalic=true, particle,
  hepprocess=true, process, unit, xref]{\ATLASLATEXPATH atlasphysics}
%\usepackage{\ATLASLATEXPATH atlasheavyion-mod}

% Files with references in BibTeX format and biblatex style options
\addbibresource{atlas_physics.bib}

\graphicspath{{../../logos/}}

% Useful macros for documentation
\newcommand{\File}[1]{\texttt{#1}\xspace}
\newcommand{\Macro}[1]{\texttt{\textbackslash #1}\xspace}
\newcommand{\Option}[1]{\textsf{#1}\xspace}
\newcommand{\Package}[1]{\texttt{#1}\xspace}

%-------------------------------------------------------------------------------
% Generic document information
%-------------------------------------------------------------------------------
% Set author and title for the PDF file
\hypersetup{pdftitle={ATLAS LaTeX guide},pdfauthor={Ian Brock}}

\AtlasTitle{Symbols defined in \File{atlasphysics.sty}}

\author{Ian C. Brock}
\affil{University of Bonn}

% \AtlasVersion{\ATPackageVersion}

\AtlasAbstract{%
  This note lists the symbols defined in \File{atlasphysics.sty}.
  These provide examples of how to define your own symbols, as well as many symbols
  that are often used in ATLAS documents.

  This document was generated using version \ATPackageVersion\ of the ATLAS \LaTeX\ package.
  The \TeX\ Live version is set to \ATTeXLiveVersion.
  It uses the option \Option{atlasstyle}, which implies that the standard ATLAS preprint style is used.
}


%-------------------------------------------------------------------------------
% This is where the document really begins
%-------------------------------------------------------------------------------
\begin{document}

\maketitle

\tableofcontents

%-------------------------------------------------------------------------------
\section{\File{atlasphysics.sty} style file}
\label{sec:atlasphysics}
%-------------------------------------------------------------------------------

The \File{atlasphysics.sty} style file implements a series of useful
shortcuts to typeset a physics paper, such as particle
symbols.

Options are parsed with the \Package{kvoptions} package, which is included by default.
The style file can included in the preamble of your paper with the usual
syntax:
%
\begin{verbatim}
  \usepackage{\ATLASLATEXPATH atlasphysics}
\end{verbatim}
%
As of version 01-00-00 the file is actually split into smaller files,
which can be included or not using options.
The following options are available, where the default setting is given in parentheses:
\begin{description}
\item[BSM](false) BSM and SUSY particles.
\item[hion](false) Useful macros for heavy ion physics.
\item[jetetmiss](false) Useful macros for Jet/ETmiss publications.
\item[journal](true) Journal abbreviations and a few other definitions for references.
\item[math](false) A few extra maths definitions.
\item[misc](true) Miscellaneous definitions that are often used.
\item[other](false) Definitions that used to be in \File{atlasphysics.sty}, 
  but are probably too specialised to be needed by most people.
\item[particle](true) Standard Model particles and some combinations.
\item[hepparticle](false) Standard Model particles and some combinations using the \Package{hepparticle} package.
  This option will supersede \Option{particle} at some time.
\item[process](false) Some example processes. 
  These are not included by default as the current choice is rather arbitrary
  and certainly not complete.
\item[hepprocess](false) Some example processes using the \Package{hepparticle} package.
  These are not included by default as the current choice is rather arbitrary
  and certainly not complete.
  This option will supersede \Option{process} at some time.
\item[unit](true) Units that used to be defined -- not needed if you use \Package{siunitx} or \Package{hepunits}.
\item[xref](true) Useful abbreviations for cross-references.
\item[texlive=YYYY](2016) Set if you use an older version of \TeX\ Live like 2009.
\item[texmf] Use the syntax \Macro{usepackage\{package\}}
  instead of \Macro{usepackage\{\textbackslash ATLASLATEXPATH package\}} to include packages.
  This is needed if you install \Package{atlaslatex} centrally,
  rather than in a \File{latex} subdirectory.
\end{description}
Note that \Option{BSM} and \Option{BSM=true} are equivalent.
Use the syntax \Option{option=false} to turn off an option.

If the  option \Option{texmf} is included, the subfiles are included using the command:
\verb|\RequirePackage{atlasparticle}| etc. instead of \verb|\RequirePackage{\ATLASLATEXPATH atlasparticle}|.
This is useful if you install the ATLAS \LaTeX\ package in a central directory such as \File{\$\{HOME\}/texmf/tex/latex}.

All definitions are done in a consistent way using \verb|\newcommand*|.
All definitions use \verb|\ensuremath| where appropriate and are terminated with
\verb|\xspace|, so you can simply write {\verb|\ttbar production| instead of
\verb|\ttbar\ production| or \verb|\ttbar{} production| to get \enquote{\ttbar production}.

The \Package{hepparticles}~\cite{hepparticles} package has uniform definitions for many Standard Model and BSM particles.
In fact you should use the package \Package{heppennames} and/or \Package{hepnicenames},
which contain many predefined particles.
These packages load \Package{hepparticles}, which can then be used to define more particles if you need them.
One very nice feature of these packages is that you can switch between italic and upright symbols via an option.

See \Sect{\ref{sec:old}} for details on changes that were introduced when
when going from version 00-04-05 of \Package{atlasnote}
to version 01-00-00 of \Package{atlaslatex}.
Let me know if you spot some other changes that are not documented here!

Changes to the contents that might affect existing documents are given in \Sect{\ref{sec:change}}.

The following sections list the macros defined in the various files.



\twocolumn
%-------------------------------------------------------------------------------
\section{\File{atlasparticle.sty}}

Turn on including these definitions with the option \Option{particle=true} and off with the option \Option{particle=false}.

As an alternative you can use the \Package{hepparticles}~\cite{hepparticles} package,
which has uniform definitions for many Standard Model and BSM particles.

Use the \Option{hepparticle=true} instead of \Option{particle=true} to use the \Package{hepparticle} definitions.

This version of the document uses the \Option{hepparticle} and \Option{hepprocess} options and so
does not show the definitions made using the \Option{particle} and \Option{process} options.
%%-------------------------------------------------------------------------------
% Useful particle definitions - include with particle option in atlasphysics.sty
% Included by default
%-------------------------------------------------------------------------------
%\let\sst=\scriptscriptstyle

% +--------------------------------------------------------------------+
% |                                                                    |
% |  Particle-antiparticle pair notations                              |
% |                                                                    |
% +--------------------------------------------------------------------+
\newcommand*{\pp}{\ensuremath{pp}\xspace}
\newcommand*{\antibar}[1]{\ensuremath{#1\bar{#1}}\xspace}
\newcommand*{\pbar}{\ensuremath{\bar{p}}\xspace}
\newcommand*{\ppbar}{\antibar{p}}
\newcommand*{\tbar}{\ensuremath{\bar{t}}\xspace}
\newcommand*{\ttbar}{\antibar{t}}
% SIunits already defines \bbar - supersede the definition
\@ifpackageloaded{SIunits}{%}
  \renewcommand*{\bbar}{\ensuremath{\bar{b}}\xspace}
}{%
  \newcommand*{\bbar}{\ensuremath{\bar{b}}\xspace}
}
\newcommand*{\bbbar}{\antibar{b}}
\newcommand*{\cbar}{\ensuremath{\bar{c}}\xspace}
\newcommand*{\ccbar}{\antibar{c}}
\newcommand*{\sbar}{\ensuremath{\bar{s}}\xspace}
\newcommand*{\ssbar}{\antibar{s}}
\newcommand*{\ubar}{\ensuremath{\bar{u}}\xspace}
\newcommand*{\uubar}{\antibar{u}}
\newcommand*{\dbar}{\ensuremath{\bar{d}}\xspace}
\newcommand*{\ddbar}{\antibar{d}}
\newcommand*{\fbar}{\ensuremath{\bar{f}}\xspace}
\newcommand*{\ffbar}{\antibar{f}}
\newcommand*{\qbar}{\ensuremath{\bar{q}}\xspace}
\newcommand*{\qqbar}{\antibar{q}}
\newcommand*{\nbar}{\ensuremath{\bar{\nu}}\xspace}
\newcommand*{\nnbar}{\antibar{\nu}}

% +--------------------------------------------------------------------+
% |                                                                    |
% |  e+e-, etc.                                                        |
% |                                                                    |
% +--------------------------------------------------------------------+
\newcommand*{\ee}{\ensuremath{e^{+} e^{-}}\xspace}
\newcommand*{\epm}{\ensuremath{e^{\pm}}\xspace}
\newcommand*{\epem}{\ensuremath{e^{+} e^-}\xspace}
\newcommand*{\mumu}{\ensuremath{\mu^{+} \mu^{-}}\xspace}
\newcommand*{\tautau}{\ensuremath{\tau^{+} \tau^{-}}\xspace}
\newcommand*{\leplep}{\ensuremath{\ell^{+} \ell^{-}}\xspace}
\newcommand*{\ellell}{\ensuremath{\ell^{+} \ell^{-}}\xspace}
\newcommand*{\enu}{\ensuremath{e \nu}\xspace}
\newcommand*{\munu}{\ensuremath{\mu \nu}\xspace}
\newcommand*{\lnu}{\ensuremath{\ell \nu}\xspace}

% +--------------------------------------------------------------------+
% |                                                                    |
% |  Vector bosons                                                     |
% |                                                                    |
% +--------------------------------------------------------------------+
\newcommand*{\Zzero}{\ensuremath{Z}\xspace}
\newcommand*{\Zboson}{\ensuremath{Z}\xspace}
\newcommand*{\Wplus}{\ensuremath{W^{+}}\xspace}
\newcommand*{\Wminus}{\ensuremath{W^{-}}\xspace}
\newcommand*{\Wboson}{\ensuremath{W}\xspace}
\newcommand*{\Wpm}{\ensuremath{W^{\pm}}\xspace}
\newcommand*{\Wmp}{\ensuremath{W^{\mp}}\xspace}

% +--------------------------------------------------------------------+
% |                                                                    |
% |  pi, pi0, pi+, pi-, pi+-, eta, eta1, etc.                          |
% |                                                                    |
% +--------------------------------------------------------------------+
%\let\pii=\pi
%\renewcommand*{\pi}{\ensuremath{\pii}\xspace}
\newcommand*{\pizero}{\ensuremath{\pi^{0}}\xspace}
\newcommand*{\piplus}{\ensuremath{\pi^{+}}\xspace}
\newcommand*{\piminus}{\ensuremath{\pi^{-}}\xspace}
\newcommand*{\pipm}{\ensuremath{\pi^{\pm}}\xspace}
\newcommand*{\pimp}{\ensuremath{\pi^{\mp}}\xspace}
%\let\etaa=\eta
%\renewcommand*{\eta}{\ensuremath{\etaa}\xspace}
\newcommand*{\etaprime}{\ensuremath{\eta^{\scriptscriptstyle\prime}}\xspace}

% +--------------------------------------------------------------------+
% |                                                                    |
% |  K0, K+, K-, K0L, K0S                                              |
% |                                                                    |
% +--------------------------------------------------------------------+
\newcommand*{\Kzero}{\ensuremath{K^{0}}\xspace}
\newcommand*{\Kzerobar}{\ensuremath{\overline{K}\vphantom{K}^{0}}\xspace}
\newcommand*{\kaon}{\ensuremath{K}\xspace}
\newcommand*{\Kplus}{\ensuremath{K^{+}}\xspace}
\newcommand*{\Kminus}{\ensuremath{K^{-}}\xspace}
\newcommand*{\KzeroL}{\ensuremath{K^{0}_{\text{L}}}\xspace}
\newcommand*{\Kzerol}{\ensuremath{K^{0}_{\text{L}}}\xspace}
\newcommand*{\Klong}{\ensuremath{K^{0}_{\text{L}}}\xspace}
\newcommand*{\KzeroS}{\ensuremath{K^{0}_{\text{S}}}\xspace}
\newcommand*{\Kzeros}{\ensuremath{K^{0}_{\text{S}}}\xspace}
\newcommand*{\Kshort}{\ensuremath{K^{0}_{\text{S}}}\xspace}
\newcommand*{\Kstar}{\ensuremath{K^{*}}\xspace}

% +--------------------------------------------------------------------+
% |                                                                    |
% |  J/psi, psi prime, etc.                                            |
% |                                                                    |
% +--------------------------------------------------------------------+
%\let\psii=\psi  %  Save normal "\psi" definition, since I redefine it.
%\renewcommand*{\psi}{\ensuremath{\psii}\xspace}
\newcommand*{\jpsi}{\ensuremath{J/\psi}\xspace}
\newcommand*{\Jpsi}{\ensuremath{J/\psi}\xspace}
%\newcommand*{\psip}{\ensuremath{\psi^{\scriptscriptstyle\prime}}\xspace}
\newcommand*{\psip}{\ensuremath{\psi(\text{2S})}\xspace}
\newcommand*{\chic}{\ensuremath{\raise.4ex\hbox{$\chi$}_{{c}}}\xspace}

% +--------------------------------------------------------------------+
% |                                                                    |
% |  Upsilons of various sorts                                         |
% |                                                                    |
% +--------------------------------------------------------------------+
% \newcommand*{\Ups}{\ensuremath{\mit\Upsilon}\xspace}
%\newcommand*{\Upsp}{\ensuremath{\mit\Upsilon^{\scriptscriptstyle\prime}}\xspace}
%\newcommand*{\Upspp}{\ensuremath{\mit\Upsilon^{\scriptscriptstyle\prime\prime}}\xspace}
%\newcommand*{\Upsppp}{\ensuremath{\mit\Upsilon^{\scriptscriptstyle\prime\prime\prime}}\xspace}
%\newcommand*{\Upspppp}{\ensuremath{\mit\Upsilon^{\scriptscriptstyle\prime\prime\prime\prime}}\xspace}
\newcommand*{\UoneS}{\ensuremath{\mit\Upsilon(\text{1S})}\xspace}
\newcommand*{\Ups}[1]{\ensuremath{\mit{\Upsilon}(\text{#1S})}\xspace}
\newcommand*{\chib}{\ensuremath{\raise.4ex\hbox{$\chi$}_{{b}}}\xspace}

% +--------------------------------------------------------------------+
% |                                                                    |
% |  C physics                                                         |
% |                                                                    |
% +--------------------------------------------------------------------+
\newcommand*{\Dstar}{\ensuremath{D^{*}}\xspace}
%\newcommand*{\Dsstar}{\ensuremath{D^{**}}\xspace}

% +--------------------------------------------------------------------+
% |                                                                    |
% |  B physics                                                         |
% |                                                                    |
% +--------------------------------------------------------------------+
\newcommand*{\Bd}{\ensuremath{B_{d}^{0}}\xspace}
\newcommand*{\Bs}{\ensuremath{B_{s}^{0}}\xspace}
\newcommand*{\Bu}{\ensuremath{B_{u}}\xspace}
\newcommand*{\Bc}{\ensuremath{B_{c}}\xspace}
\newcommand*{\Lb}{\ensuremath{\Lambda_{b}}\xspace}
\newcommand*{\Bstar}{\ensuremath{B^{*}}\xspace}
\newcommand*{\BoBo}{\ensuremath{B^{0}\mbox{--}\bar{B}^{0}}\xspace}
\newcommand*{\BodBod}{\ensuremath{B^{0}_{d}\mbox{--}\bar{B}^{0}_{d}}\xspace}
\newcommand*{\BosBos}{\ensuremath{B^{0}_{s}\mbox{--}\bar{B}^{0}_{s}}\xspace}


A generic macro \verb|\Ups[1]| is available.
It is defined such that \verb|\Ups{3}| produces \Ups{3}.


\newpage
%-------------------------------------------------------------------------------
\section{\File{atlashepparticle.sty}}

Turn on including these definitions with the option \Option{hepparticle=true} and off with the option \Option{hepparticle=false}.

These definitions use the \Package{hepparticles}~\cite{hepparticles} package,
which has uniform definitions for many Standard Model and BSM particles.
The names used are those in \Package{heppennames}.
The package loads \Package{hepparticles}, which can then be used to define more particles if you need them.
One very nice feature of these packages is that you can switch between italic and upright symbols via an option.

%-------------------------------------------------------------------------------
% Useful particle definitions - include with hepparticle option in atlasphysics.sty
% Not included by default
% Use particle definitions from the heppparticle/heppennames package.
% Note that atlasparticle and atlashepparticle are mutually exclusive!
%-------------------------------------------------------------------------------
%\let\sst=\scriptscriptstyle

% Make sure heppennames is loaded.
\ifAPHY@hepitalic
  \RequirePackage[italic]{heppennames}
\else
  \RequirePackage{heppennames}
\fi

% Adjust the kerning for bar over particles with narrow glyphs (from Andy Buckley)
\def\@shiftlen@anti@gen@bar{0mu}

% +--------------------------------------------------------------------+
% |                                                                    |
% |  Particle-antiparticle pair notations                              |
% |                                                                    |
% +--------------------------------------------------------------------+
\newcommand*{\pp}{\Pp{}\Pp}
%\newcommand*{\antibar}[1]{\ensuremath{#1\bar{#1}}\xspace}
\newcommand*{\pbar}{\Pap}
\newcommand*{\ppbar}{\Pp{}\Pap}
\newcommand*{\tbar}{\Paqt}
\newcommand*{\ttbar}{\Pqt{}\Paqt}
% SIunits already defines \bbar - supersede the definition
\@ifpackageloaded{SIunits}{%
  \renewcommand*{\bbar}{\Paqb}
}{%
  \newcommand*{\bbar}{\Paqb}
}
\newcommand*{\bbbar}{\Pqb{}\Paqb}
\newcommand*{\cbar}{\Paqc}
\newcommand*{\ccbar}{\Pqc{}\Paqc}
\newcommand*{\sbar}{\Paqs}
\newcommand*{\ssbar}{\Pqs{}\Paqs}
\newcommand*{\ubar}{\Paqu}
\newcommand*{\uubar}{\Pqu{}\Paqu}
\newcommand*{\dbar}{\Paqd}
\newcommand*{\ddbar}{\Pqd{}\Paqd}
\newcommand*{\fbar}{\Paf}
\newcommand*{\ffbar}{\Pf{}\Paf}
\newcommand*{\qbar}{\Paq}
\newcommand*{\qqbar}{\Pq{}\Paq}
\newcommand*{\nbar}{\Pagn}
\newcommand*{\nnbar}{\Pgn{}\Pagn}

% +--------------------------------------------------------------------+
% |                                                                    |
% |  e+e-, etc.                                                        |
% |                                                                    |
% +--------------------------------------------------------------------+
\newcommand*{\ee}{\Pep{}\Pem}
\newcommand*{\epm}{\Pepm}
\newcommand*{\epem}{\Pep{}\Pem}
\newcommand*{\mumu}{\Pgmp{}\Pgmm}
\newcommand*{\tautau}{\Pgtp{}\Pgtm}
\newcommand*{\leplep}{\Plp{}\Plm}
\newcommand*{\ellell}{\Plp{}\Plm}
\newcommand*{\enu}{\Pe{}\Pgn}
\newcommand*{\munu}{\Pgm{}\Pgn}
\newcommand*{\lnu}{\Pl{}\Pgn}

% +--------------------------------------------------------------------+
% |                                                                    |
% |  Vector bosons                                                     |
% |                                                                    |
% +--------------------------------------------------------------------+
\newcommand*{\Zzero}{\PZ}
\newcommand*{\Zboson}{\PZ}
\newcommand*{\Wplus}{\PWp}
\newcommand*{\Wminus}{\PWm}
\newcommand*{\Wboson}{\PW}
\newcommand*{\Wpm}{\PWpm}
\newcommand*{\Wmp}{\PWmp}

% +--------------------------------------------------------------------+
% |                                                                    |
% |  pi, pi0, pi+, pi-, pi+-, eta, eta1, etc.                          |
% |                                                                    |
% +--------------------------------------------------------------------+
%\let\pii=\pi
%\renewcommand*{\pi}{\ensuremath{\pii}\xspace}
\newcommand*{\pizero}{\Pgpz}
\newcommand*{\piplus}{\Pgpp}
\newcommand*{\piminus}{\Pgpm}
\newcommand*{\pipm}{\Pgppm}
\newcommand*{\pimp}{\Pgpmp}
%\let\etaa=\eta
%\renewcommand*{\eta}{\ensuremath{\etaa}\xspace}
\newcommand*{\etaprime}{\Pghpr}

% +--------------------------------------------------------------------+
% |                                                                    |
% |  K0, K+, K-, K0L, K0S                                              |
% |                                                                    |
% +--------------------------------------------------------------------+
\newcommand*{\Kzero}{\PKz}
\newcommand*{\Kzerobar}{\PaKz}
\newcommand*{\kaon}{\PK}
\newcommand*{\Kplus}{\PKp}
\newcommand*{\Kminus}{\PKm}
\newcommand*{\KzeroL}{\PKzL}
\newcommand*{\Kzerol}{\PKzL}
\newcommand*{\Klong}{\PKzL}
\newcommand*{\KzeroS}{\PKzS}
\newcommand*{\Kzeros}{\PKzS}
\newcommand*{\Kshort}{\PKzS}
\newcommand*{\Kstar}{\PKst}

% +--------------------------------------------------------------------+
% |                                                                    |
% |  J/psi, psi prime, etc.                                            |
% |                                                                    |
% +--------------------------------------------------------------------+
%\let\psii=\psi  %  Save normal "\psi" definition, since I redefine it.
%\renewcommand*{\psi}{\ensuremath{\psii}\xspace}
\newcommand*{\jpsi}{\PJgy}
\newcommand*{\Jpsi}{\PJgy}
\newcommand*{\psip}{\Pgyii}
\newcommand*{\chic}{\Pcgc}

% +--------------------------------------------------------------------+
% |                                                                    |
% |  Upsilons of various sorts                                         |
% |                                                                    |
% +--------------------------------------------------------------------+
% \newcommand*{\Ups}{\ensuremath{\mit\Upsilon}\xspace}
%\newcommand*{\Upsp}{\ensuremath{\mit\Upsilon^{\scriptscriptstyle\prime}}\xspace}
%\newcommand*{\Upspp}{\ensuremath{\mit\Upsilon^{\scriptscriptstyle\prime\prime}}\xspace}
%\newcommand*{\Upsppp}{\ensuremath{\mit\Upsilon^{\scriptscriptstyle\prime\prime\prime}}\xspace}
%\newcommand*{\Upspppp}{\ensuremath{\mit\Upsilon^{\scriptscriptstyle\prime\prime\prime\prime}}\xspace}
\newcommand*{\UoneS}{\PgUi}
\newcommand*{\Ups}[1]{\ensuremath{\PgU(\text{#1S})}\xspace}
\DeclareRobustCommand{\Pbgc}{\HepParticle{\chi}{b}{}\xspace}
\newcommand*{\chib}{\Pbgc}

% +--------------------------------------------------------------------+
% |                                                                    |
% |  C physics                                                         |
% |                                                                    |
% +--------------------------------------------------------------------+
\newcommand*{\Dstar}{\PDst}
%\newcommand*{\Dsstar}{\ensuremath{D^{**}}\xspace}

% +--------------------------------------------------------------------+
% |                                                                    |
% |  B physics                                                         |
% |                                                                    |
% +--------------------------------------------------------------------+
\newcommand*{\Bd}{\PdB}
\newcommand*{\Bs}{\PsB}
\newcommand*{\Bu}{\ensuremath{\PB_{\Pqu}}}
\newcommand*{\Bc}{\ensuremath{\PB_{\Pqc}}}
\newcommand*{\Lb}{\PbgL}
\newcommand*{\Bstar}{\PBst}
\newcommand*{\BoBo}{\PBz{}\mbox{--}\PaBz}
\newcommand*{\BodBod}{\PdB{}\mbox{--}\PadB}
\newcommand*{\BosBos}{\PsB{}\mbox{--}\PasB}


A generic macro \verb|\Ups[1]| is available.
It is defined such that \verb|\Ups{3}| produces \Ups{3}.


%-------------------------------------------------------------------------------
\newpage
%-------------------------------------------------------------------------------
\section{\File{atlasjournal.sty}}

Turn on including these definitions with the option \Option{journal=true} and off with the option \Option{journal=false}.

%
% +--------------------------------------------------------------------+
% |                                                                    |
% |  Journals                                                          |
% |                                                                    |
% +--------------------------------------------------------------------+
%
\newcommand*{\AcPA}{Acta Phys. Austriaca}
\newcommand*{\ARevNS}{Ann. Rev.\ Nucl.\ Sci.{}}
\newcommand*{\CPC}{Comp.\ Phys.\ Comm.{}}
\newcommand*{\FortP}{Fortschr.\ Phys.{}}
\newcommand*{\IJMP}{Int.\ J.\ Mod.\ Phys.{}}
\newcommand*{\JETP}{Sov.\ Phys.\ JETP}
\newcommand*{\JETPL}{JETP Lett.{}}
\newcommand*{\JaFi}{Jad.\ Fiz.{}}
\newcommand*{\JMP}{J.\ Math.\ Phys.{}}
\newcommand*{\MPL}{Mod.\ Phys.\ Lett.{}}
\newcommand*{\NCim}{Nuovo Cimento}
\newcommand*{\NIM}{Nucl.\ Instrum.\ Meth.{}}
\newcommand*{\NP}{Nucl.\ Phys.{}}
\newcommand*{\NPB}{Nucl.\ Phys.\ B}
\newcommand*{\PL}{Phys.\ Lett.{}}
\newcommand*{\PLB}{Phys.\ Lett.\ B}
\newcommand*{\PR}{Phys.\ Rev.{}}
\newcommand*{\PRC}{Phys.\ Rev.\ C}
\newcommand*{\PRD}{Phys.\ Rev.\ D}
\newcommand*{\PRL}{Phys.\ Rev.\ Lett.{}}
\newcommand*{\PRep}{Phys.\ Rep.{}}
\newcommand*{\RMP}{Rev.\ Mod.\ Phys.{}}
\newcommand*{\ZfP}{Z.\ Phys.{}}
\newcommand*{\EPJ}{Eur.\ Phys.\ J.{}}
\newcommand*{\EPJC}{Eur.\ Phys.\ J.\ C}
%
% +--------------------------------------------------------------------+
% |                                                                    |
% |  Useful abbreviations in references                                |
% |                                                                    |
% +--------------------------------------------------------------------+
%
\newcommand*{\collab}{Collaboration}



\newpage
%-------------------------------------------------------------------------------
\section{\File{atlasmisc.sty}}

Turn on including these definitions with the option \Option{misc=true} and off with the option \Option{misc=false}.

\begin{xtabular}{ll}
\verb|\pT| & \pT \\
\verb|\pt| & \pt \\
\verb|\ET| & \ET \\
\verb|\eT| & \eT \\
\verb|\et| & \et \\
\verb|\HT| & \HT \\
\verb|\pTsq| & \pTsq \\
\verb|\MET| & \MET \\
\verb|\met| & \met \\
\verb|\sumET| & \sumET \\
\verb|\EjetRec| & \EjetRec \\
\verb|\PjetRec| & \PjetRec \\
\verb|\EjetTru| & \EjetTru \\
\verb|\PjetTru| & \PjetTru \\
\verb|\EjetDM| & \EjetDM \\
\verb|\Rcone| & \Rcone \\
\verb|\abseta| & \abseta \\
\verb|\Ecm| & \Ecm \\
\verb|\rts| & \rts \\
\verb|\sqs| & \sqs \\
\verb|\Nevt| & \Nevt \\
\verb|\zvtx| & \zvtx \\
\verb|\dzero| & \dzero \\
\verb|\zzsth| & \zzsth \\
\verb|\RunOne| & \RunOne \\
\verb|\RunTwo| & \RunTwo \\
\verb|\RunThr| & \RunThr \\
\verb|\kt| & \kt \\
\verb|\antikt| & \antikt \\
\verb|\Antikt| & \Antikt \\
\verb|\btag| & \btag \\
\verb|\btagged| & \btagged \\
\verb|\bquark| & \bquark \\
\verb|\bquarks| & \bquarks \\
\verb|\bjet| & \bjet \\
\verb|\bjets| & \bjets \\
\verb|\mh| & \mh \\
\verb|\mW| & \mW \\
\verb|\mZ| & \mZ \\
\verb|\mH| & \mH \\
\verb|\ACERMC| & \ACERMC \\
\verb|\ALPGEN| & \ALPGEN \\
\verb|\GEANT| & \GEANT \\
\verb|\Herwigpp| & \Herwigpp \\
\verb|\HERWIGpp| & \HERWIGpp \\
\verb|\Herwig| & \Herwig \\
\verb|\HERWIG| & \HERWIG \\
\verb|\JIMMY| & \JIMMY \\
\verb|\MADSPIN| & \MADSPIN \\
\verb|\MADGRAPH| & \MADGRAPH \\
\verb|\MGMCatNLO| & \MGMCatNLO \\
\verb|\MCatNLO| & \MCatNLO \\
\verb|\AMCatNLO| & \AMCatNLO \\
\verb|\MCFM| & \MCFM \\
\verb|\METOP| & \METOP \\
\verb|\POWHEG| & \POWHEG \\
\verb|\POWHEGBOX| & \POWHEGBOX \\
\verb|\POWPYTHIA| & \POWPYTHIA \\
\verb|\PROTOS| & \PROTOS \\
\verb|\PYTHIA| & \PYTHIA \\
\verb|\SHERPA| & \SHERPA \\
\verb|\Comphep| & \Comphep \\
\verb|\Perugia| & \Perugia \\
\verb|\Prospino| & \Prospino \\
\verb|\LO| & \LO \\
\verb|\NLO| & \NLO \\
\verb|\NLL| & \NLL \\
\verb|\NNLO| & \NNLO \\
\verb|\muF| & \muF \\
\verb|\muR| & \muR \\
\verb|\ra| & \ra \\
\verb|\la| & \la \\
\verb|\rarrow| & \rarrow \\
\verb|\larrow| & \larrow \\
\verb|\lapprox| & \lapprox \\
\verb|\rapprox| & \rapprox \\
\verb|\gam| & \gam \\
\verb|\stat| & \stat \\
\verb|\syst| & \syst \\
\verb|\radlength| & \radlength \\
\verb|\StoB| & \StoB \\
\verb|\alphas| & \alphas \\
\verb|\NF| & \NF \\
\verb|\NC| & \NC \\
\verb|\CF| & \CF \\
\verb|\CA| & \CA \\
\verb|\TF| & \TF \\
\verb|\Lms| & \Lms \\
\verb|\Lmsfive| & \Lmsfive \\
\verb|\KT| & \KT \\
\verb|\Vcb| & \Vcb \\
\verb|\Vub| & \Vub \\
\verb|\Vtd| & \Vtd \\
\verb|\Vts| & \Vts \\
\verb|\Vtb| & \Vtb \\
\verb|\Vcs| & \Vcs \\
\verb|\Vud| & \Vud \\
\verb|\Vus| & \Vus \\
\verb|\Vcd| & \Vcd \\
\end{xtabular}


\noindent A length \Macro{figwidth} is defined that is \SI{2}{\cm} smaller than \Macro{textwidth}.

\noindent Most Monte Carlo generators also have a form with a suffix \enquote{V}
that allows you to include the version, e.g.
\verb|\PYTHIAV{8}| to produce \PYTHIAV{8} or
\verb|\PYTHIAV{8 (v8.160)}| to produce \PYTHIAV{8 (v8.160)}.

\noindent A generic macro \verb|\twomass| is defined, so that for example
\verb|\twomass{\mu}{\mu}| produces \twomass{\mu}{\mu} and \verb|\twomass{\mu}{e}| produces \twomass{\mu}{e}.

A macro \verb|\dk| is also defined which makes it easier to write down decay chains.
For example
\begin{verbatim}
\[\eqalign{a \to & b+c\\
   & \dk & e+f \\
   && \dk g+h}
\]
\end{verbatim}
produces
\[\eqalign{a \to & b+c\cr
   & \dk & e+f \cr
   && \dk g+h}
\]
Note that \Macro{eqalign} is also redefined in this package so that \Macro{dk} works.

The following macro names have been changed:\\
\verb|\ptsq| $\to$ \verb|\pTsq|.


\newpage
%-------------------------------------------------------------------------------
\section{\File{atlasxref.sty}}

Turn on including these definitions with the option \Option{xref=true} and off with the option \Option{xref=false}.

\begin{xtabular}{ll}
\end{xtabular}


\noindent The following macros with arguments are also defined:
\begin{xtabular}{ll}
\verb|\App{1}|  & \App{1}\\
\verb|\Eqn{1}|  & \Eqn{1}\\
\verb|\Fig{1}|  & \Fig{1}\\
\verb|\Ref{1}|  & \Ref{1}\\
\verb|\Sect{1}| & \Sect{1}\\
\verb|\Tab{1}|  & \Tab{1}\\
\verb|\Apps{1}{4}| & \Apps{1}{4} \\
\verb|\Eqns{1}{4}| & \Eqns{1}{4} \\
\verb|\Figs{1}{4}| & \Figs{1}{4} \\
\verb|\Refs{1}{4}| & \Refs{1}{4} \\
\verb|\Sects{1}{4}| & \Sects{1}{4} \\
\verb|\Tabs{1}{4}| & \Tabs{1}{4} \\
\verb|\Apprange{1}{4}| & \Apprange{1}{4} \\
\verb|\Eqnrange{1}{4}| & \Eqnrange{1}{4} \\
\verb|\Figrange{1}{4}| & \Figrange{1}{4} \\
\verb|\Refrange{1}{4}| & \Refrange{1}{4} \\
\verb|\Sectrange{1}{4}| & \Sectrange{1}{4} \\
\verb|\Tabrange{1}{4}| & \Tabrange{1}{4}
\end{xtabular}

The idea is that you can adapt these definitions according to your own preferences (or those of a journal).


\newpage
%-------------------------------------------------------------------------------
\section{\File{atlasbsm.sty}}

Turn on including these definitions with the option \Option{BSM} and off with the option \Option{BSM=false}.

The macro \Macro{susy} simply puts a tilde ($\tilde{\ }$) over its argument,
e.g.\ \verb|\susy{q}| produces \susy{q}.

For \susy{q}, \susy{t}, \susy{b}, \slepton, \sel, \smu and
\stau, L and R states are defined; for stop, sbottom and stau also the
light (1) and heavy (2) states.
There are four neutralinos and two charginos defined, 
the index number unfortunately needs to be written out completely. 
For the charginos the last letter(s) indicate(s) the charge: 
\enquote{p} for $+$, \enquote{m} for $-$, and \enquote{pm} for $\pm$.

%
% +--------------------------------------------------------------------+
% |                                                                    |
% |  New particle stuff                                                |
% |                                                                    |
% +--------------------------------------------------------------------+
%
\newcommand*{\Azero}{\ensuremath{A^0}\xspace}
\newcommand*{\hzero}{\ensuremath{h^0}\xspace}
\newcommand*{\Hzero}{\ensuremath{H^0}\xspace}
\newcommand*{\Hboson}{\ensuremath{H}\xspace}
\newcommand*{\Hplus}{\ensuremath{H^+}\xspace}
\newcommand*{\Hminus}{\ensuremath{H^-}\xspace}
\newcommand*{\Hpm}{\ensuremath{H^{\pm}}\xspace}
\newcommand*{\Hmp}{\ensuremath{H^{\mp}}\xspace}
\newcommand*{\susy}[1]{\ensuremath{\tilde{#1}}\xspace}
\newcommand*{\ggino}{\ensuremath{\mathchoice%
      {\displaystyle\raise.4ex\hbox{$\displaystyle\tilde\chi$}}%
         {\textstyle\raise.4ex\hbox{$\textstyle\tilde\chi$}}%
       {\scriptstyle\raise.3ex\hbox{$\scriptstyle\tilde\chi$}}%
 {\scriptscriptstyle\raise.3ex\hbox{$\scriptscriptstyle\tilde\chi$}}}\xspace}

\newcommand*{\chinop}{\ensuremath{\mathchoice%
      {\displaystyle\raise.4ex\hbox{$\displaystyle\tilde\chi^+$}}%
         {\textstyle\raise.4ex\hbox{$\textstyle\tilde\chi^+$}}%
       {\scriptstyle\raise.3ex\hbox{$\scriptstyle\tilde\chi^+$}}%
 {\scriptscriptstyle\raise.3ex\hbox{$\scriptscriptstyle\tilde\chi^+$}}}\xspace}
\newcommand*{\chinom}{\ensuremath{\mathchoice%
      {\displaystyle\raise.4ex\hbox{$\displaystyle\tilde\chi^-$}}%
         {\textstyle\raise.4ex\hbox{$\textstyle\tilde\chi^-$}}%
       {\scriptstyle\raise.3ex\hbox{$\scriptstyle\tilde\chi^-$}}%
 {\scriptscriptstyle\raise.3ex\hbox{$\scriptscriptstyle\tilde\chi^-$}}}\xspace}
\newcommand*{\chinopm}{\ensuremath{\mathchoice%
      {\displaystyle\raise.4ex\hbox{$\displaystyle\tilde\chi^\pm$}}%
         {\textstyle\raise.4ex\hbox{$\textstyle\tilde\chi^\pm$}}%
       {\scriptstyle\raise.3ex\hbox{$\scriptstyle\tilde\chi^\pm$}}%
 {\scriptscriptstyle\raise.3ex\hbox{$\scriptscriptstyle\tilde\chi^\pm$}}}\xspace}
\newcommand*{\chinomp}{\ensuremath{\mathchoice%
      {\displaystyle\raise.4ex\hbox{$\displaystyle\tilde\chi^\mp$}}%
         {\textstyle\raise.4ex\hbox{$\textstyle\tilde\chi^\mp$}}%
       {\scriptstyle\raise.3ex\hbox{$\scriptstyle\tilde\chi^\mp$}}%
 {\scriptscriptstyle\raise.3ex\hbox{$\scriptscriptstyle\tilde\chi^\mp$}}}\xspace}

\newcommand*{\chinoonep}{\ensuremath{\mathchoice%
      {\displaystyle\raise.4ex\hbox{$\displaystyle\tilde\chi^+_1$}}%
         {\textstyle\raise.4ex\hbox{$\textstyle\tilde\chi^+_1$}}%
       {\scriptstyle\raise.3ex\hbox{$\scriptstyle\tilde\chi^+_1$}}%
 {\scriptscriptstyle\raise.3ex\hbox{$\scriptscriptstyle\tilde\chi^+_1$}}}\xspace}
\newcommand*{\chinoonem}{\ensuremath{\mathchoice%
      {\displaystyle\raise.4ex\hbox{$\displaystyle\tilde\chi^-_1$}}%
         {\textstyle\raise.4ex\hbox{$\textstyle\tilde\chi^-_1$}}%
       {\scriptstyle\raise.3ex\hbox{$\scriptstyle\tilde\chi^-_1$}}%
 {\scriptscriptstyle\raise.3ex\hbox{$\scriptscriptstyle\tilde\chi^-_1$}}}\xspace}
\newcommand*{\chinoonepm}{\ensuremath{\mathchoice%
      {\displaystyle\raise.4ex\hbox{$\displaystyle\tilde\chi^\pm_1$}}%
         {\textstyle\raise.4ex\hbox{$\textstyle\tilde\chi^\pm_1$}}%
       {\scriptstyle\raise.3ex\hbox{$\scriptstyle\tilde\chi^\pm_1$}}%
 {\scriptscriptstyle\raise.3ex\hbox{$\scriptscriptstyle\tilde\chi^\pm_1$}}}\xspace}

\newcommand*{\chinotwop}{\ensuremath{\mathchoice%
      {\displaystyle\raise.4ex\hbox{$\displaystyle\tilde\chi^+_2$}}%
         {\textstyle\raise.4ex\hbox{$\textstyle\tilde\chi^+_2$}}%
       {\scriptstyle\raise.3ex\hbox{$\scriptstyle\tilde\chi^+_2$}}%
 {\scriptscriptstyle\raise.3ex\hbox{$\scriptscriptstyle\tilde\chi^+_2$}}}\xspace}
\newcommand*{\chinotwom}{\ensuremath{\mathchoice%
      {\displaystyle\raise.4ex\hbox{$\displaystyle\tilde\chi^-_2$}}%
         {\textstyle\raise.4ex\hbox{$\textstyle\tilde\chi^-_2$}}%
       {\scriptstyle\raise.3ex\hbox{$\scriptstyle\tilde\chi^-_2$}}%
 {\scriptscriptstyle\raise.3ex\hbox{$\scriptscriptstyle\tilde\chi^-_2$}}}\xspace}
\newcommand*{\chinotwopm}{\ensuremath{\mathchoice%
      {\displaystyle\raise.4ex\hbox{$\displaystyle\tilde\chi^\pm_2$}}%
         {\textstyle\raise.4ex\hbox{$\textstyle\tilde\chi^\pm_2$}}%
       {\scriptstyle\raise.3ex\hbox{$\scriptstyle\tilde\chi^\pm_2$}}%
 {\scriptscriptstyle\raise.3ex\hbox{$\scriptscriptstyle\tilde\chi^\pm_2$}}}\xspace}

\newcommand*{\nino}{\ensuremath{\mathchoice%
      {\displaystyle\raise.4ex\hbox{$\displaystyle\tilde\chi^0$}}%
         {\textstyle\raise.4ex\hbox{$\textstyle\tilde\chi^0$}}%
       {\scriptstyle\raise.3ex\hbox{$\scriptstyle\tilde\chi^0$}}%
 {\scriptscriptstyle\raise.3ex\hbox{$\scriptscriptstyle\tilde\chi^0$}}}\xspace}

\newcommand*{\ninoone}{\ensuremath{\mathchoice%
      {\displaystyle\raise.4ex\hbox{$\displaystyle\tilde\chi^0_1$}}%
         {\textstyle\raise.4ex\hbox{$\textstyle\tilde\chi^0_1$}}%
       {\scriptstyle\raise.3ex\hbox{$\scriptstyle\tilde\chi^0_1$}}%
 {\scriptscriptstyle\raise.3ex\hbox{$\scriptscriptstyle\tilde\chi^0_1$}}}\xspace}
\newcommand*{\ninotwo}{\ensuremath{\mathchoice%
      {\displaystyle\raise.4ex\hbox{$\displaystyle\tilde\chi^0_2$}}%
         {\textstyle\raise.4ex\hbox{$\textstyle\tilde\chi^0_2$}}%
       {\scriptstyle\raise.3ex\hbox{$\scriptstyle\tilde\chi^0_2$}}%
 {\scriptscriptstyle\raise.3ex\hbox{$\scriptscriptstyle\tilde\chi^0_2$}}}\xspace}
\newcommand*{\ninothree}{\ensuremath{\mathchoice%
      {\displaystyle\raise.4ex\hbox{$\displaystyle\tilde\chi^0_3$}}%
         {\textstyle\raise.4ex\hbox{$\textstyle\tilde\chi^0_3$}}%
       {\scriptstyle\raise.3ex\hbox{$\scriptstyle\tilde\chi^0_3$}}%
 {\scriptscriptstyle\raise.3ex\hbox{$\scriptscriptstyle\tilde\chi^0_3$}}}\xspace}
\newcommand*{\ninofour}{\ensuremath{\mathchoice%
      {\displaystyle\raise.4ex\hbox{$\displaystyle\tilde\chi^0_4$}}%
         {\textstyle\raise.4ex\hbox{$\textstyle\tilde\chi^0_4$}}%
       {\scriptstyle\raise.3ex\hbox{$\scriptstyle\tilde\chi^0_4$}}%
 {\scriptscriptstyle\raise.3ex\hbox{$\scriptscriptstyle\tilde\chi^0_4$}}}\xspace}

\newcommand*{\gravino}{\ensuremath{\tilde{G}}\xspace}
\newcommand*{\Zprime}{\ensuremath{Z^\prime}\xspace}
\newcommand*{\Zstar}{\ensuremath{Z^{*}}\xspace}
\newcommand*{\squark}{\ensuremath{\tilde{q}}\xspace}
\newcommand*{\squarkL}{\ensuremath{\tilde{q}_{\mathrm{L}}}\xspace}
\newcommand*{\squarkR}{\ensuremath{\tilde{q}_{\mathrm{R}}}\xspace}
\newcommand*{\gluino}{\ensuremath{\tilde{g}}\xspace}
\renewcommand*{\stop}{\ensuremath{\tilde{t}}\xspace}
\newcommand*{\stopone}{\ensuremath{\tilde{t}_1}\xspace}
\newcommand*{\stoptwo}{\ensuremath{\tilde{t}_2}\xspace}
\newcommand*{\stopL}{\ensuremath{\tilde{t}_{\mathrm{L}}}\xspace}
\newcommand*{\stopR}{\ensuremath{\tilde{t}_{\mathrm{R}}}\xspace}
\newcommand*{\sbottom}{\ensuremath{\tilde{b}}\xspace}
\newcommand*{\sbottomone}{\ensuremath{\tilde{b}_1}\xspace}
\newcommand*{\sbottomtwo}{\ensuremath{\tilde{b}_2}\xspace}
\newcommand*{\sbottomL}{\ensuremath{\tilde{b}_{\mathrm{L}}}\xspace}
\newcommand*{\sbottomR}{\ensuremath{\tilde{b}_{\mathrm{R}}}\xspace}
\newcommand*{\slepton}{\ensuremath{\tilde{\ell}}\xspace}
\newcommand*{\sleptonL}{\ensuremath{\tilde{\ell}_{\mathrm{L}}}\xspace}
\newcommand*{\sleptonR}{\ensuremath{\tilde{\ell}_{\mathrm{R}}}\xspace}
\newcommand*{\sel}{\ensuremath{\tilde{e}}\xspace}
\newcommand*{\selL}{\ensuremath{\tilde{e}_{\mathrm{L}}}\xspace}
\newcommand*{\selR}{\ensuremath{\tilde{e}_{\mathrm{R}}}\xspace}
\newcommand*{\smu}{\ensuremath{\tilde{\mu}}\xspace}
\newcommand*{\smuL}{\ensuremath{\tilde{\mu}_{\mathrm{L}}}\xspace}
\newcommand*{\smuR}{\ensuremath{\tilde{\mu}_{\mathrm{R}}}\xspace}
\newcommand*{\stau}{\ensuremath{\tilde{\tau}}\xspace}
\newcommand*{\stauL}{\ensuremath{\tilde{\tau}_{\mathrm{L}}}\xspace}
\newcommand*{\stauR}{\ensuremath{\tilde{\tau}_{\mathrm{R}}}\xspace}
\newcommand*{\stauone}{\ensuremath{\tilde{\tau}_1}\xspace}
\newcommand*{\stautwo}{\ensuremath{\tilde{\tau}_2}\xspace}
\newcommand*{\snu}{\ensuremath{\tilde{\nu}}\xspace}



\newpage
%-------------------------------------------------------------------------------
\section{\File{atlasheavyion.sty}}

Turn on including these definitions with the option \Option{hion=true} and off with the option \Option{hion=false}.
The heavy ion definitions use the package \Package{mhchem} to help with the formatting of chemical elements.
This package is included by \File{atlasheavyion.sty}.

\begin{xtabular}{ll}
\verb|\NucNuc| & \NucNuc \\
\verb|\nn| & \nn \\
\verb|\pn| & \pn \\
\verb|\np| & \np \\
\verb|\PbPb| & \PbPb \\
\verb|\AuAu| & \AuAu \\
\verb|\CuCu| & \CuCu \\
\verb|\pNuc| & \pNuc \\
\verb|\pdA| & \pdA \\
\verb|\dAu| & \dAu \\
\verb|\pPb| & \pPb \\
\verb|\Npart| & \Npart \\
\verb|\avgNpart| & \avgNpart \\
\verb|\Ncoll| & \Ncoll \\
\verb|\avgNcoll| & \avgNcoll \\
\verb|\TA| & \TA \\
\verb|\avgTA| & \avgTA \\
\verb|\TPb| & \TPb \\
\verb|\avgTPb| & \avgTPb \\
\verb|\TAA| & \TAA \\
\verb|\avgTAA| & \avgTAA \\
\verb|\TAB| & \TAB \\
\verb|\avgTAB| & \avgTAB \\
\verb|\TpPb| & \TpPb \\
\verb|\avgTpPb| & \avgTpPb \\
\verb|\Gl| & \Gl \\
\verb|\GG| & \GG \\
\verb|\sqn| & \sqn \\
\verb|\lns| & \lns \\
\verb|\sumETPb| & \sumETPb \\
\verb|\sumETp| & \sumETp \\
\verb|\sumETA| & \sumETA \\
\verb|\RAA| & \RAA \\
\verb|\RCP| & \RCP \\
\verb|\RpA| & \RpA \\
\verb|\RpPb| & \RpPb \\
\verb|\dif| & \dif \\
\verb|\dNchdeta| & \dNchdeta \\
\verb|\dNevtdET| & \dNevtdET \\
\verb|\ystar| & \ystar \\
\verb|\ycms| & \ycms \\
\verb|\ygappb| & \ygappb \\
\verb|\ygapp| & \ygapp \\
\verb|\fgap| & \fgap \\
\end{xtabular}


%The following symbols were removed or modified with respect to the original submission
%\input{atlasheavyion-mod}


\newpage
%-------------------------------------------------------------------------------
\section{\File{atlasjetetmiss.sty}}

Turn on including these definitions with the option \Option{jetetmiss=true} and off with the option \Option{jetetmiss=false}.

\begin{xtabular}{ll}
\verb|\topo| & \topo \\
\verb|\Topo| & \Topo \\
\verb|\topos| & \topos \\
\verb|\Topos| & \Topos \\
\verb|\pileup| & \pileup \\
\verb|\Pileup| & \Pileup \\
\verb|\insitu| & \insitu \\
\verb|\Insitu| & \Insitu \\
\verb|\LS| & \LS \\
\verb|\NLOjet| & \NLOjet \\
\verb|\Fastjet| & \Fastjet \\
\verb|\TwoToTwo| & \TwoToTwo \\
\verb|\largeR| & \largeR \\
\verb|\LargeR| & \LargeR \\
\verb|\akt| & \akt \\
\verb|\Akt| & \Akt \\
\verb|\AKT| & \AKT \\
\verb|\AKTFat| & \AKTFat \\
\verb|\AKTPrune| & \AKTPrune \\
\verb|\AKTFilt| & \AKTFilt \\
\verb|\KTSix| & \KTSix \\
\verb|\ca| & \ca \\
\verb|\CamKt| & \CamKt \\
\verb|\CASix| & \CASix \\
\verb|\CAFat| & \CAFat \\
\verb|\CAPrune| & \CAPrune \\
\verb|\CAFilt| & \CAFilt \\
\verb|\Rfilt| & \Rfilt \\
\verb|\ymin| & \ymin \\
\verb|\fcut| & \fcut \\
\verb|\Rsub| & \Rsub \\
\verb|\mufrac| & \mufrac \\
\verb|\Rcut| & \Rcut \\
\verb|\zcut| & \zcut \\
\verb|\ftile| & \ftile \\
\verb|\fem| & \fem \\
\verb|\fpres| & \fpres \\
\verb|\fhec| & \fhec \\
\verb|\ffcal| & \ffcal \\
\verb|\central| & \central \\
\verb|\ecap| & \ecap \\
\verb|\forward| & \forward \\
\verb|\Npv| & \Npv \\
\verb|\Nref| & \Nref \\
\verb|\Navg| & \Navg \\
\verb|\avgmu| & \avgmu \\
\verb|\JES| & \JES \\
\verb|\JMS| & \JMS \\
\verb|\EMJES| & \EMJES \\
\verb|\GCWJES| & \GCWJES \\
\verb|\LCWJES| & \LCWJES \\
\verb|\EM| & \EM \\
\verb|\GCW| & \GCW \\
\verb|\LCW| & \LCW \\
\verb|\GSL| & \GSL \\
\verb|\GS| & \GS \\
\verb|\MTF| & \MTF \\
\verb|\MPF| & \MPF \\
\verb|\Njet| & \Njet \\
\verb|\njet| & \njet \\
\verb|\ETjet| & \ETjet \\
\verb|\etjet| & \etjet \\
\verb|\pTavg| & \pTavg \\
\verb|\ptavg| & \ptavg \\
\verb|\pTjet| & \pTjet \\
\verb|\ptjet| & \ptjet \\
\verb|\pTcorr| & \pTcorr \\
\verb|\ptcorr| & \ptcorr \\
\verb|\pTjeti| & \pTjeti \\
\verb|\ptjeti| & \ptjeti \\
\verb|\pTrecoil| & \pTrecoil \\
\verb|\ptrecoil| & \ptrecoil \\
\verb|\pTleading| & \pTleading \\
\verb|\ptleading| & \ptleading \\
\verb|\pTjetEM| & \pTjetEM \\
\verb|\ptjetEM| & \ptjetEM \\
\verb|\pThat| & \pThat \\
\verb|\pthat| & \pthat \\
\verb|\pTprobe| & \pTprobe \\
\verb|\ptprobe| & \ptprobe \\
\verb|\pTref| & \pTref \\
\verb|\ptref| & \ptref \\
\verb|\pToff| & \pToff \\
\verb|\ptoff| & \ptoff \\
\verb|\pToffjet| & \pToffjet \\
\verb|\ptoffjet| & \ptoffjet \\
\verb|\pTZ| & \pTZ \\
\verb|\ptZ| & \ptZ \\
\verb|\pTtrue| & \pTtrue \\
\verb|\pttrue| & \pttrue \\
\verb|\pTtruth| & \pTtruth \\
\verb|\pttruth| & \pttruth \\
\verb|\pTreco| & \pTreco \\
\verb|\ptreco| & \ptreco \\
\verb|\pTtrk| & \pTtrk \\
\verb|\pttrk| & \pttrk \\
\verb|\ptrk| & \ptrk \\
\verb|\pTtrkjet| & \pTtrkjet \\
\verb|\pttrkjet| & \pttrkjet \\
\verb|\ntrk| & \ntrk \\
\verb|\EoverP| & \EoverP \\
\verb|\Etrue| & \Etrue \\
\verb|\Etruth| & \Etruth \\
\verb|\Ecalo| & \Ecalo \\
\verb|\EcaloEM| & \EcaloEM \\
\verb|\asym| & \asym \\
\verb|\Response| & \Response \\
\verb|\Rcalo| & \Rcalo \\
\verb|\RcaloM| & \RcaloM \\
\verb|\RcaloEM| & \RcaloEM \\
\verb|\RMPF| & \RMPF \\
\verb|\EcaloCALIB| & \EcaloCALIB \\
\verb|\RcaloCALIB| & \RcaloCALIB \\
\verb|\EcaloEMJES| & \EcaloEMJES \\
\verb|\RcaloEMJES| & \RcaloEMJES \\
\verb|\EcaloGCWJES| & \EcaloGCWJES \\
\verb|\RcaloGCWJES| & \RcaloGCWJES \\
\verb|\EcaloLCWJES| & \EcaloLCWJES \\
\verb|\RcaloLCWJES| & \RcaloLCWJES \\
\verb|\Rtrack| & \Rtrack \\
\verb|\rtrk| & \rtrk \\
\verb|\Rtrk| & \Rtrk \\
\verb|\rtrackjet| & \rtrackjet \\
\verb|\rtrackjetiso| & \rtrackjetiso \\
\verb|\rtrackjetnoniso| & \rtrackjetnoniso \\
\verb|\rtrackjetisoratio| & \rtrackjetisoratio \\
\verb|\gammajet| & \gammajet \\
\verb|\deltaphijetgamma| & \deltaphijetgamma \\
\verb|\rapjet| & \rapjet \\
\verb|\etajet| & \etajet \\
\verb|\phijet| & \phijet \\
\verb|\etadet| & \etadet \\
\verb|\etatrk| & \etatrk \\
\verb|\Rmin| & \Rmin \\
\verb|\DeltaR| & \DeltaR \\
\verb|\DetaDphi| & \DetaDphi \\
\verb|\Deta| & \Deta \\
\verb|\Drap| & \Drap \\
\verb|\DetaOneTwo| & \DetaOneTwo \\
\verb|\DyDphi| & \DyDphi \\
\verb|\DeltaRdef| & \DeltaRdef \\
\verb|\DeltaRydef| & \DeltaRydef \\
\verb|\DeltaRtrk| & \DeltaRtrk \\
\verb|\JVF| & \JVF \\
\verb|\cJVF| & \cJVF \\
\verb|\RpT| & \RpT \\
\verb|\JVT| & \JVT \\
\verb|\ghostpt| & \ghostpt \\
\verb|\ghostptavg| & \ghostptavg \\
\verb|\ghostfm| & \ghostfm \\
\verb|\ghostfmi| & \ghostfmi \\
\verb|\ghostdensity| & \ghostdensity \\
\verb|\ghostrho| & \ghostrho \\
\verb|\Aghost| & \Aghost \\
\verb|\Amu| & \Amu \\
\verb|\Amui| & \Amui \\
\verb|\jetarea| & \jetarea \\
\verb|\jetareafm| & \jetareafm \\
\verb|\jetareai| & \jetareai \\
\verb|\Rkt| & \Rkt \\
\verb|\pTmuslope| & \pTmuslope \\
\verb|\ptmuslope| & \ptmuslope \\
\verb|\pTnpvslope| & \pTnpvslope \\
\verb|\ptnpvslope| & \ptnpvslope \\
\verb|\pTmuunc| & \pTmuunc \\
\verb|\ptmuunc| & \ptmuunc \\
\verb|\pTnpvunc| & \pTnpvunc \\
\verb|\ptnpvunc| & \ptnpvunc \\
\verb|\sumPt| & \sumPt \\
\verb|\sumpt| & \sumpt \\
\verb|\sumpTtrk| & \sumpTtrk \\
\verb|\sumpttrk| & \sumpttrk \\
\verb|\nPUtrk| & \nPUtrk \\
\verb|\mjet| & \mjet \\
\verb|\mlead| & \mlead \\
\verb|\mleadavg| & \mleadavg \\
\verb|\Mjet| & \Mjet \\
\verb|\massjet| & \massjet \\
\verb|\masscorr| & \masscorr \\
\verb|\mthresh| & \mthresh \\
\verb|\mjetavg| & \mjetavg \\
\verb|\masstrkjet| & \masstrkjet \\
\verb|\width| & \width \\
\verb|\wcalo| & \wcalo \\
\verb|\wtrk| & \wtrk \\
\verb|\shapeV| & \shapeV \\
\verb|\pTsubjet| & \pTsubjet \\
\verb|\ptsubjet| & \ptsubjet \\
\verb|\sjone| & \sjone \\
\verb|\sjtwo| & \sjtwo \\
\verb|\msubjone| & \msubjone \\
\verb|\msubjtwo| & \msubjtwo \\
\verb|\pTsubji| & \pTsubji \\
\verb|\ptsubji| & \ptsubji \\
\verb|\pTsubjone| & \pTsubjone \\
\verb|\ptsubjone| & \ptsubjone \\
\verb|\pTsubjtwo| & \pTsubjtwo \\
\verb|\ptsubjtwo| & \ptsubjtwo \\
\verb|\Rsubjets| & \Rsubjets \\
\verb|\DRsubjets| & \DRsubjets \\
\verb|\yij| & \yij \\
\verb|\dcut| & \dcut \\
\verb|\dmin| & \dmin \\
\verb|\dij| & \dij \\
\verb|\Dij| & \Dij \\
\verb|\Donetwo| & \Donetwo \\
\verb|\Dtwothr| & \Dtwothr \\
\verb|\yonetwo| & \yonetwo \\
\verb|\ytwothr| & \ytwothr \\
\verb|\yonetwoDef| & \yonetwoDef \\
\verb|\ytwothrDef| & \ytwothrDef \\
\verb|\xj| & \xj \\
\verb|\jetFunc| & \jetFunc \\
\verb|\tauone| & \tauone \\
\verb|\tautwo| & \tautwo \\
\verb|\tauthr| & \tauthr \\
\verb|\tauN| & \tauN \\
\verb|\tautwoone| & \tautwoone \\
\verb|\tauthrtwo| & \tauthrtwo \\
\verb|\dip| & \dip \\
\verb|\diponetwo| & \diponetwo \\
\verb|\diptwothr| & \diptwothr \\
\verb|\diponethr| & \diponethr \\
\end{xtabular}


\noindent The macro \Macro{etaRange} produces what you would expect:
\verb|\etaRange{-2.5}{+2.5}| produces \etaRange{-2.5}{+2.5} while
\verb|\AetaRange{1.0}| produces \AetaRange{1.0}.
The macro \Macro{avg} can be used for average values:
\verb|\avg{\mu}| produces \avg{\mu}.


\newpage
%-------------------------------------------------------------------------------
\section{\File{atlasmath.sty}}

Turn on including these definitions with the option \Option{math=true} and off with the option \Option{math=false}.

\begin{xtabular}{ll}
\verb|\boxsq| & \boxsq \\
\verb|\grad| & \grad \\
\end{xtabular}


\noindent The macro \Macro{spinor} is also defined.
\verb|\spinor{u}| produces \spinor{u}.


\newpage
%-------------------------------------------------------------------------------
\section{\File{atlasother.sty}}

Turn on including these definitions with the option \Option{other} and off with the option \Option{other=false}.

%-------------------------------------------------------------------------------
% Other definitions that used to be in atlasphysics.sty.
% Include with other option in atlasphysics.sty.
% Not included by default.
% Also needs atlasmisc.sty (option misc) and
% Also needs atlasparticle.sty (option particle).
%-------------------------------------------------------------------------------

\chardef\letterchar=11
\chardef\otherchar=12
\chardef\eolinechar=5

\newcommand*{\etpt}{\ensuremath{1/p_{\text{T}} - 1/E_{\text{T}}}\xspace}
\newcommand*{\etptsig}{\ensuremath{(1/p_{\text{T}} - 1/E_{\text{T}})/(\sigma(1/p_{\text{T}}))}\xspace}
\newcommand*{\begL}{\ensuremath{\SI{E31}{\per\cm\squared\per\second}}\xspace}
\newcommand*{\lowL}{\ensuremath{\SI{E33}{\per\cm\squared\per\second}}\xspace}
\newcommand*{\highL}{\ensuremath{\SI{E34}{\per\cm\squared\per\second}}\xspace}

\newcommand*{\Epsb}{\ensuremath{\epsilon_{b}}\xspace}
\newcommand*{\Epsc}{\ensuremath{\epsilon_{c}}\xspace} % Subscript italic not roman (EE)

% Conflicts with \mA in siunitx version 1
\providecommand*{\mA}{\ensuremath{m_{A}}\xspace}

% +--------------------------------------------------------------------+
% |                                                                    |
% |  sin2thetaW m_W m_Z etc.                                           |
% |                                                                    |
% +--------------------------------------------------------------------+
\newcommand*{\Mtau}{\ensuremath{m_{\tau}}\xspace}
\newcommand*{\swsq}{\ensuremath{\sin^2\!\theta_{W}}\xspace}
\newcommand*{\swel}{\ensuremath{\sin^2\!\theta_{\text{eff}}^{\text{lept}}}\xspace}
\newcommand*{\swsqb}{\ensuremath{\sin^2\!\overline{\theta}_{W}}\xspace}
\newcommand*{\swsqon}{\ensuremath{\swsq\equiv 1-\mW^2/\mZ^2}\xspace}
\newcommand*{\gv}{\ensuremath{g_{\text{V}}}\xspace}
\newcommand*{\ga}{\ensuremath{g_{\text{A}}}\xspace}
\newcommand*{\gvbar}{\ensuremath{\bar{g}_\text{V}}\xspace}
\newcommand*{\gabar}{\ensuremath{\bar{g}_\text{A}}\xspace}

% +--------------------------------------------------------------------+
% |                                                                    |
% |  Useful Z0 type stuff    Gammas, asymmetries                       |
% |                                                                    |
% +--------------------------------------------------------------------+
\newcommand*{\Zzv}{\ensuremath{Z^{\textstyle *}}\xspace}
\newcommand*{\Abb}{\ensuremath{A_{\bbbar}}\xspace}
\newcommand*{\Acc}{\ensuremath{A_{\ccbar}}\xspace}
\newcommand*{\Aqq}{\ensuremath{A_{\qqbar}}\xspace}
\newcommand*{\Afb}{\ensuremath{A_{{fb}}}\xspace}
\newcommand*{\GZ}{\ensuremath{\Gamma_{Z}}\xspace}
\newcommand*{\GW}{\ensuremath{\Gamma_{W}}\xspace}
\newcommand*{\GH}{\ensuremath{\Gamma_{H}}\xspace}
\newcommand*{\GamHad}{\ensuremath{\Gamma_{\text{had}}}\xspace}
\newcommand*{\Gbb}{\ensuremath{\Gamma_{\bbbar}}\xspace}
\newcommand*{\Rbb}{\ensuremath{R_{\bbbar}}\xspace}
\newcommand*{\Gcc}{\ensuremath{\Gamma_{\ccbar}}\xspace}
\newcommand*{\Gvis}{\ensuremath{\Gamma_{\text{vis}}}\xspace}
\newcommand*{\Ginv}{\ensuremath{\Gamma_{\text{inv}}}\xspace}

% +--------------------------------------------------------------------+
% |                                                                    |
% |  Notation for the P-lines: \nspj211 -->  2 1P1, etc.               |
% |                                                                    |
% +--------------------------------------------------------------------+
\def\nsPj#1#2#3{\ensuremath{#1\,^{#2}\!P_{#3}}}%
\let\nspj=\nsPj
\def\nsSj#1#2#3{\ensuremath{#1\,^{#2}\!S_{#3}}}%
\let\nssj=\nsSj


%-------------------------------------------------------------------------------

\newpage
%-------------------------------------------------------------------------------
\section{\File{atlasprocess.sty}}

Turn on including these definitions with the option \Option{process} and off with the option \Option{process=false}.

As an alternative you can use the \Package{hepparticles}~\cite{hepparticles} package,
which has uniform definitions for many Standard Model and BSM particles.
Use the \Option{hepprocess=true} instead of \Option{process=true} to use the \Package{hepparticle} definitions.

This version of the document uses the \Option{hepparticle} and \Option{hepprocess} options and so
does not show the definitions made using the \Option{particle} and \Option{process} options.
%%-------------------------------------------------------------------------------
% Definitions of miscellaneous processes.
% Include with process option in atlasphysics.sty.
% Not included by default.
% Also needs atlasparticle.sty (option particle).
%-------------------------------------------------------------------------------

\newcommand*{\btol}{\ensuremath{b \rightarrow \ell}\xspace}
\newcommand*{\ctol}{\ensuremath{c \rightarrow \ell}\xspace}
\newcommand*{\btoctol}{\ensuremath{b \rightarrow c \rightarrow \ell}\xspace}
\newcommand*{\Jee}{\ensuremath{\Jpsi \rightarrow \epem}\xspace}
\newcommand*{\Jmm}{\ensuremath{\Jpsi \rightarrow \mumu}\xspace}
\newcommand*{\Jmumu}{\ensuremath{\Jpsi \rightarrow \mumu}\xspace}
\newcommand*{\Wjj}{\ensuremath{W \rightarrow jj}\xspace}
\newcommand*{\tjjb}{\ensuremath{t \rightarrow jjb}\xspace}
\newcommand*{\Hbb}{\ensuremath{H \rightarrow b\bar b}\xspace}
\newcommand*{\Hgg}{\ensuremath{H \rightarrow \gamma\gamma}\xspace}
\newcommand*{\Hllll}{\ensuremath{H \rightarrow \ell\ell\ell\ell}\xspace}
\newcommand*{\Hmmmm}{\ensuremath{H \rightarrow \mu\mu\mu\mu}\xspace}
\newcommand*{\Heeee}{\ensuremath{H \rightarrow eeee}\xspace}
\newcommand*{\Zll}{\ensuremath{Z \rightarrow \ell\ell}\xspace}
\newcommand*{\Zlplm}{\ensuremath{Z \rightarrow \leplep}\xspace}
\newcommand*{\Zee}{\ensuremath{Z \rightarrow ee}\xspace}
\newcommand*{\Zepem}{\ensuremath{Z \rightarrow \epem}\xspace}
\newcommand*{\Zmm}{\ensuremath{Z \rightarrow \mu\mu}\xspace}
\newcommand*{\Zmpmm}{\ensuremath{Z \rightarrow \mumu}\xspace}
\newcommand*{\Ztt}{\ensuremath{Z \rightarrow \tau\tau}\xspace}
\newcommand*{\Ztptm}{\ensuremath{Z \rightarrow \tautau}\xspace}
\newcommand*{\Zbb}{\ensuremath{Z \rightarrow \bbbar}\xspace}
\newcommand*{\Wln}{\ensuremath{W \rightarrow \ell\nu}\xspace}
\newcommand*{\Wen}{\ensuremath{W \rightarrow e\nu}\xspace}
\newcommand*{\Wmn}{\ensuremath{W \rightarrow \mu\nu}\xspace}
\newcommand*{\Wlnu}{\ensuremath{W \rightarrow \lnu}\xspace}
\newcommand*{\Wenu}{\ensuremath{W \rightarrow \enu}\xspace}
\newcommand*{\Wmunu}{\ensuremath{W \rightarrow \munu}\xspace}
\newcommand*{\Wqqbar}{\ensuremath{W \rightarrow \qqbar}\xspace}
\newcommand*{\Amm}{\ensuremath{A \rightarrow \mu\mu}\xspace}
\newcommand*{\Ztautau}{\ensuremath{Z \rightarrow \tau\tau}\xspace}
\newcommand*{\Wtaunu}{\ensuremath{W \rightarrow \tau\nu}\xspace}
\newcommand*{\Atautau}{\ensuremath{A \rightarrow \tau\tau}\xspace}
\newcommand*{\Htautau}{\ensuremath{H \rightarrow \tau\tau}\xspace}

\newcommand*{\tWb}{\ensuremath{t \rightarrow Wb}\xspace}

% W+jets without the \, before the + looks a bit better
\newcommand*{\Wjets}{\ensuremath{\Wboson\text{+\,jets}}\xspace}
\newcommand*{\Zjets}{\ensuremath{\Zboson\text{\,+\,jets}}\xspace}

\newcommand*{\Brjl}{\ensuremath{\mathrm{Br}(\Jpsi \rightarrow \leplep)}}



\newpage
%-------------------------------------------------------------------------------
\section{\File{atlashepprocess.sty}}

Turn on including these definitions with the option \Option{hepprocess} and off with the option \Option{hepprocess=false}.

The packages \Package{heppennames} and/or \Package{hepnicenames} contain many predefined particles,
so you do not need to define them yourself.
These packages load \Package{hepparticles}, which can then be used to define more particles if you need them.
One very nice feature of these packages is that you can switch between italic and upright symbols via an option.

\begin{xtabular}{ll}
\verb|\btol| & \btol \\
\verb|\ctol| & \ctol \\
\verb|\btoctol| & \btoctol \\
\verb|\Jee| & \Jee \\
\verb|\Jmm| & \Jmm \\
\verb|\Jmumu| & \Jmumu \\
\verb|\Wjj| & \Wjj \\
\verb|\tjjb| & \tjjb \\
\verb|\Hbb| & \Hbb \\
\verb|\Hgg| & \Hgg \\
\verb|\Hllll| & \Hllll \\
\verb|\Hmmmm| & \Hmmmm \\
\verb|\Heeee| & \Heeee \\
\verb|\Zll| & \Zll \\
\verb|\Zlplm| & \Zlplm \\
\verb|\Zee| & \Zee \\
\verb|\Zepem| & \Zepem \\
\verb|\Zmm| & \Zmm \\
\verb|\Zmpmm| & \Zmpmm \\
\verb|\Ztt| & \Ztt \\
\verb|\Ztptm| & \Ztptm \\
\verb|\Zbb| & \Zbb \\
\verb|\Wln| & \Wln \\
\verb|\Wen| & \Wen \\
\verb|\Wmn| & \Wmn \\
\verb|\Wlnu| & \Wlnu \\
\verb|\Wenu| & \Wenu \\
\verb|\Wmunu| & \Wmunu \\
\verb|\Wqqbar| & \Wqqbar \\
\verb|\Amm| & \Amm \\
\verb|\Ztautau| & \Ztautau \\
\verb|\Wtaunu| & \Wtaunu \\
\verb|\Atautau| & \Atautau \\
\verb|\Htautau| & \Htautau \\
\verb|\tWb| & \tWb \\
\verb|\Wjets| & \Wjets \\
\verb|\Zjets| & \Zjets \\
\verb|\Brjl| & \Brjl \\
\end{xtabular}



%-------------------------------------------------------------------------------
\newpage
%-------------------------------------------------------------------------------
\section{\File{atlasunit.sty}}

Turn on including these definitions with the option \Option{unit} and off with the option \Option{unit=false}.

\begin{xtabular}{ll}
\verb|\TeV| & \TeV \\
\verb|\GeV| & \GeV \\
\verb|\MeV| & \MeV \\
\verb|\keV| & \keV \\
\verb|\eV| & \eV \\
\verb|\TeV| & \TeV \\
\verb|\GeV| & \GeV \\
\verb|\MeV| & \MeV \\
\verb|\keV| & \keV \\
\verb|\eV| & \eV \\
\verb|\TeVc| & \TeVc \\
\verb|\GeVc| & \GeVc \\
\verb|\MeVc| & \MeVc \\
\verb|\keVc| & \keVc \\
\verb|\eVc| & \eVc \\
\verb|\TeVcc| & \TeVcc \\
\verb|\GeVcc| & \GeVcc \\
\verb|\MeVcc| & \MeVcc \\
\verb|\keVcc| & \keVcc \\
\verb|\eVcc| & \eVcc \\
\verb|\ifb| & \ifb \\
\verb|\ipb| & \ipb \\
\verb|\inb| & \inb \\
\verb|\degr| & \degr \\
\verb|\TeVc| & \TeVc \\
\verb|\GeVc| & \GeVc \\
\verb|\MeVc| & \MeVc \\
\verb|\keVc| & \keVc \\
\verb|\eVc| & \eVc \\
\verb|\TeVcc| & \TeVcc \\
\verb|\GeVcc| & \GeVcc \\
\verb|\MeVcc| & \MeVcc \\
\verb|\keVcc| & \keVcc \\
\verb|\eVcc| & \eVcc \\
\verb|\ifb| & \ifb \\
\verb|\ipb| & \ipb \\
\verb|\inb| & \inb \\
\verb|\degr| & \degr \\
\verb|\TeVc| & \TeVc \\
\verb|\GeVc| & \GeVc \\
\verb|\MeVc| & \MeVc \\
\verb|\keVc| & \keVc \\
\verb|\eVc| & \eVc \\
\verb|\TeVcc| & \TeVcc \\
\verb|\GeVcc| & \GeVcc \\
\verb|\MeVcc| & \MeVcc \\
\verb|\keVcc| & \keVcc \\
\verb|\eVcc| & \eVcc \\
\verb|\ifb| & \ifb \\
\verb|\ipb| & \ipb \\
\verb|\inb| & \inb \\
\verb|\degr| & \degr \\
\end{xtabular}


\noindent Lower case versions of the units also exist, e.g.\ \verb|\tev|, \verb|\gev|, \verb|\mev|, \verb|\kev|, and
\verb|\ev|. 

As mentioned above, it is highly recommended to use a units package instead of
these definitions. \Package{siunitx} is the preferred package; a good alternative is \Package{hepunits}.
If either of these packages are used \File{atlasunit.sty} is not needed.

Most units that are needed in ATLAS documents are already defined by \Package{siunitx} or
are defined in \File{atlaspackage.sty}.
A selection of them is given below.
In order to use them in your document the unit should be included in
\verb|\si| or \verb|\SI|:\\
\begin{xtabular}{ll}
\verb|\si{\TeV}| & \si{\TeV} \\
\verb|\si{\GeV}| & \si{\GeV} \\
\verb|\si{\MeV}| & \si{\MeV} \\
\verb|\si{\keV}| & \si{\keV} \\
\verb|\si{\eV}|  & \si{\eV} \\
\verb|\si{\TeVc}| & \si{\TeVc} \\
\verb|\si{\GeVc}| & \si{\GeVc} \\
\verb|\si{\MeVc}| & \si{\MeVc} \\
\verb|\si{\keVc}| & \si{\keVc} \\
\verb|\si{\eVc}|  & \si{\eVc} \\
\verb|\si{\TeVcc}| & \si{\TeVcc} \\
\verb|\si{\GeVcc}| & \si{\GeVcc} \\
\verb|\si{\MeVcc}| & \si{\MeVcc} \\
\verb|\si{\keVcc}| & \si{\keVcc} \\
\verb|\si{\eVcc}|  & \si{\eVcc} \\
\verb|\si{\nb}| & \si{\nb} \\
\verb|\si{\pb}| & \si{\pb} \\
\verb|\si{\fb}| & \si{\fb} \\
\verb|\si{\per\fb}| & \si{\per\fb} \\
\verb|\si{\per\pb}| & \si{\per\pb} \\
\verb|\si{\per\nb}| & \si{\per\nb} \\
\verb|\si{\ifb}| & \si{\ifb} \\
\verb|\si{\ipb}| & \si{\ipb} \\
\verb|\si{\inb}| & \si{\inb} \\
\verb|\si{\Hz}|  & \si{\Hz} \\
\verb|\si{\kHz}| & \si{\kHz} \\
\verb|\si{\MHz}| & \si{\MHz} \\
\verb|\si{\GHz}| & \si{\GHz} \\
\verb|\si{\degr}| & \si{\degr} \\
\verb|\si{\m}| & \si{\m} \\
\verb|\si{\cm}| & \si{\cm} \\
\verb|\si{\mm}| & \si{\mm} \\
\verb|\si{\um}| & \si{\um} \\
\verb|\si{\micron}| & \si{\micron} \\
\end{xtabular}

%-------------------------------------------------------------------------------

%-------------------------------------------------------------------------------
\onecolumn
%-------------------------------------------------------------------------------
\section{Changes}
\label{sec:change}
%-------------------------------------------------------------------------------

\textbf{Version 01-08-01} of \Package{atlaslatex} includes quite a few definitions from the Jet/ETmiss group.
A new style file has been created \File{atlasjetetmiss.sty} that is not included by default.
Some of the definitions from the Jet/ETmiss group are of more general use and so have been merged into existing style files:
\begin{description}
\item[atlasmisc.sty] List of Monte Carlo generators expanded:
  \Macro{POWHEGBOX}, \Macro{POWPYTHIA}.
  Add MC macros with suffix \enquote{V} for version number.
  \Macro{kt}, \Macro{antikt}, \Macro{Antikt}, \Macro{LO}, \Macro{NLO}, \Macro{NLL}, \Macro{NNLO},
  \Macro{muF}, \Macro{muR}.
  Added macros \Macro{Runone}, \Macro{Runtwo}, \Macro{Runthr},
  Added \Macro{radlength} and \Macro{StoB}.
  Added some standard $b$-tagging terms:
  \Macro{btag}, \Macro{btagged}, \Macro{bquark}, \Macro{bquarks}, \Macro{bjet}, \Macro{bjets}.
\item[atlasparticle.sty] Now includes \Macro{pp}, \Macro{enu}, \Macro{munu},
\item[atlasprocess.sty] Added \Macro{Zbb}, \Macro{Ztt},
  \Macro{Zlplm}, \Macro{Zepem}, \Macro{Zmpmm}, \Macro{Ztptm},
  \Macro{tWb}, \Macro{Wqqbar},
  \Macro{Wlnu}, \Macro{Wenu}, \Macro{Wmunu},
  \Macro{Wjets}, \Macro{Zjets}.
  The definition of \Macro{Hllll} was corrected.
\item[atlasheavyion.sty] \Macro{pp} moved to \File{atlasparticle.sty}.
\end{description}

This version also introduced the (optional) use of the \Package{heppennames} package.
The style files \File{atlashepparticle.sty} and \File{atlashepprocess.sty}
are intended to replace \File{atlasparticle.sty} and \File{atlasprocess.sty}.
Several particle definitions were removed from the \Package{atlasparticle} package,
as they just enable a few Greek letters: $\pi$, $\eta$ and $\psi$ to be used directly in text mode.
In addition, the primed $\Upsilon$ resonances, e.g.\ $\Upsilon''$,
as well as $D^{**}$ were removed.
as the official names are \Ups{3} etc.,

The definitions of \MeV, \GeV\ etc.\ in \texttt{atlasunit.sty} were updated in order to remove the \texttt{if} tests in them.
The \texttt{if} tests caused a problem in a paper draft, although the reason was not understood.
The new definitions do not introduce extra space before the unit in math mode.


%-------------------------------------------------------------------------------
\section{Old macros}
\label{sec:old}
%-------------------------------------------------------------------------------

With the introduction of \Package{atlaslatex} several macro names have been changed to make them more consistent.
A few have been removed. The changes include:
\begin{itemize}
\item Kaons now have a capital \enquote{K} in the macro name, e.g.\ \verb|\Kplus| for \Kplus;
\item \verb|\Ztau|, \verb|\Wtau|, \verb|\Htau| \verb|\Atau| have been replaced by
  \verb|\Ztautau|, \verb|\Wtautau|, \verb|\Htautau| \verb|\Atautau|;
\item \verb|\Ups| replaces \verb|\ups|;
  the use of \verb|\ups| to produce $\Upsilon$ in text mode has been removed;
\item \verb|\cm| has been removed, as it was the only length unit defined for text and math mode;
\item \verb|\mass| has been removed, as \verb|\twomass| can do the same thing and the name is more intuitive;
\item \verb|\mA| has been removed as it conflicts with \Package{siunitx} Version 1, which uses the name
  for milliamp.
\item \Macro{mathcal} rather than \Macro{mathscr} is recommended for luminosity and aplanarity.
\end{itemize}

Quite a few macros are more related to \Zboson physics than they are to LHC physics and have
been moved to the \File{atlasother.sty} file, which is not included by default.
There are also macros for various decay processes, \File{atlasprocess.sty} which are not included by default,
but may be useful for how you can define your favourite process.

It used to be the case that you had to use \verb|\MET{}| rather than just \verb|\MET| to get the spacing right,
as somehow \Package{xspace} did not do a good job for \met.
However, with the latest version of the packages both forms work fine.
You can compare \MET and \MET\ and see that the spacing is correct in both cases.

%-------------------------------------------------------------------------------


\printbibliography

\end{document}

%-------------------------------------------------------------------------------
% This is a template ATLAS Paper that contains suggestions and hints on how
% to get your paper draft into a form that minimises the amount of work needed
% to get it approved by the collaboration - assuming that the physics is OK!
%-------------------------------------------------------------------------------
% Specify where ATLAS LaTeX style files can be found.
\newcommand*{\ATLASLATEXPATH}{../../latex/}
%-------------------------------------------------------------------------------
\documentclass[PAPER,UKenglish,texlive=2016]{\ATLASLATEXPATH atlasdoc}

% Standard packages that can and should be used in ATLAS papers
% Use biblatex and biber for the bibliography
\usepackage[backend=biber]{\ATLASLATEXPATH atlaspackage}
% Style file with biblatex options for ATLAS documents
\usepackage{\ATLASLATEXPATH atlasbiblatex}

% Useful macros
\usepackage{\ATLASLATEXPATH atlasphysics}

% Files with references in BibTeX format and biblatex style options
\addbibresource{atlas_paper.bib}
\addbibresource{../../bibtex/bib/ATLAS.bib}
\addbibresource{../atlas_latex.bib}

% Paths for figures - do not forget the / at the end of the directory name
\graphicspath{{../../logos/}{figures/}}

%-------------------------------------------------------------------------------
% Generic document information
%-------------------------------------------------------------------------------
% Set author and title for the PDF file
\hypersetup{pdftitle={ATLAS paper template},pdfauthor={ATLAS Publication Committee}}

\AtlasTitle{General guidelines for ATLAS papers}

\author{ATLAS Publication Committee}

%\AtlasVersion{1.0}

\AtlasAbstract{%
  This document gives general guidelines on how an ATLAS paper should be structured
  and what authors should pay attention to.
  It is part of the ATLAS \LaTeX\ package which can be downloaded from
  Publication Committee TWiki pages. 
  The technical details about how to use the
  ATLAS \LaTeX\ style files are included in a separate document.

  This document was generated using version \ATPackageVersion\ of the ATLAS \LaTeX\ package.
  The \TeX\ Live version is set to \ATTeXLiveVersion.
  It uses the option \Option{atlasstyle}, which implies that the standard ATLAS preprint style is used.
}

%-------------------------------------------------------------------------------
% Extra definitions - it is usually better to collect these in a separate file
%-------------------------------------------------------------------------------
% Shorthand for \phantom to use in tables
\newcommand*{\pho}{\phantom{0}}
\newcommand{\BibTeX}{\textsc{Bib\TeX}}
\newcommand{\Macro}[1]{\texttt{\textbackslash #1}\xspace}
\newcommand{\Option}[1]{\textsf{#1}\xspace}
\newcommand{\Package}[1]{\texttt{#1}\xspace}
\newcommand{\File}[1]{\texttt{#1}\xspace}


%-------------------------------------------------------------------------------
% This is where the document really begins
%-------------------------------------------------------------------------------
\begin{document}

\maketitle

\tableofcontents
\clearpage


%-------------------------------------------------------------------------------
\section{Front page}
\label{sec:front}
%-------------------------------------------------------------------------------

Before we get to the meat of the document, a few remarks about the paper front page.


%-------------------------------------------------------------------------------
\subsection{Title} 
\label{sec:title}
%-------------------------------------------------------------------------------

The title should be concise, clear and descriptive, and should include the word \enquote{ATLAS}.


%-------------------------------------------------------------------------------
\subsection{Author list}
\label{sec:author}
%-------------------------------------------------------------------------------

The author list will be provided by the Authorship Committee and Physics Office. 
It will be created by the Physics Office at the time of the first draft announcement by PubCom, 
and added to the CDS record for the paper as a separate document from the paper itself.
Requests for exceptional authorship or for removal of names should be made to the Spokesperson and Authorship Committee while the first draft is circulating.
The author list will be archived when the second draft announcement is made by Publication Committee, 
and again added to the CDS record as a separate document from the paper. 
The front page of the paper should name \enquote{The ATLAS Collaboration} as the author.


%-------------------------------------------------------------------------------
\subsection{Abstract}
\label{sec:abstract}
%-------------------------------------------------------------------------------

The abstract should also be clear, descriptive and concise. 
It should consist of a single paragraph, occupying no more than half a page. 
It should stand on its own and the main text of the paper should not depend on it. 
The abstract should state what was measured; where the measurement was made and with what dataset / luminosity;
what method was used; 
what are the primary results and main conclusions. 
Citations, footnotes, references etc.\ should be avoided in the abstract. 
If only Monte Carlo data are used in the publication, this fact should be stated explicitly in the abstract.  


\subsubsection*{Example of a Search Analysis Abstract}

A search for the Standard Model Higgs boson in the two photon decay channel is reported, 
using \SI{1.08}{\per\fb} of proton--proton collision data at a centre-of-mass energy of \SI{7}{\TeV} recorded by the ATLAS detector.
No significant excess is observed in the investigated mass range of \SIrange{110}{150}{\GeV}. 
Upper limits on the cross-section times branching ratio of between 2.0 and 5.8 times the Standard Model prediction are derived for this mass range.

\noindent(From \enquote{Search for the Standard Model Higgs boson in the two photon decay channel with the ATLAS detector at the LHC})

\subsubsection*{Example of a Standard Model Measurement Analysis Abstract}

This Letter presents a measurement of the $WW$ production cross section in $\sqrt{s} = \SI{7}{\TeV}$ $pp$ collisions by the ATLAS experiment, using \SI{34}{\per\pb} of integrated luminosity produced by the Large Hadron Collider at CERN. Selecting events with two isolated leptons, each either an electron or a muon, 8 candidate events are observed with an expected background of $1.7 \pm 0.6$ events.
The measured cross section is 
\ifthenelse{\ATTeXLiveVersion < 2011}{%
$(41^{+20}_{-16}(\text{stat.})\pm 5(\text{syst.})\pm 1(\text{lumi.}))\;\si{\pb}$,
}{%
\SI[parse-numbers=false]{(41^{+20}_{-16}(\text{stat.})\pm 5(\text{syst.})\pm 1(\text{lumi.}))}{\pb},
}
which is consistent with the standard model prediction of \SI{44(3)}{\pb} calculated at next-to-leading order in QCD. 
In fact, the ATLAS convention is to write this as
\ifthenelse{\ATTeXLiveVersion < 2011}{%
\SI[trapambigerr=false]{44(3)}{\pb}, i.e.\ without the enclosing parentheses.
}{%
\SI[multi-part-units=single]{44(3)}{\pb}, i.e.\ without the enclosing parentheses.
}

\noindent(From \enquote{Measurement of the WW cross section in $\sqrt{s} = \SI{7}{\TeV}$ $pp$ collisions with ATLAS})


%-------------------------------------------------------------------------------
\section{Introduction}
\label{sec:intro}
%-------------------------------------------------------------------------------

This should be fairly brief. 
State the measurement being made, motivate its importance experimentally and theoretically.
It is considered bad form to point out being the first to publish something as a motivation.
Include a summary of what has already been published on this subject,
as well as the most appropriate theory references.
For a longer paper, give a brief outline of the paper, especially if the structure is unusual.
Even for a long paper, the introduction should generally not exceed one page.

Your first action before embarking on writing a paper should be to
read the ATLAS Publication Policy~\cite{publication-policy}, available
from the web pages of the Publication Committee. Chapter~3 of this
document details the refereeing and approval procedures that you will
need to follow; Chapter~4 gives information on the style.

At some point during the writing of an ATLAS paper, you should decide what journal it will be submitted to.
Please keep in mind that each journal makes specific demands on length and style.
Most journals use regular capitalisation for titles 
(i.e.\ capitalise the first word and all proper nouns, 
e.g.\ \enquote{The search for large extra dimensions}). 
The notable exception is Physical Review Letters, which capitalises the
first word and all other important words,
e.g.\ \enquote{The Search for Large Extra Dimensions}.

This document can be found on the web pages of the ATLAS Publication Committee
and in the \texttt{atlaslatex} package. 
The Publication Committee provides standard texts giving examples
for the description of the ATLAS detector, and the appropriate references,
as well as other text snippets that may be useful to avoid too much use of ATLAS jargon.
The ATLAS detector description is included in Appendix~\ref{app:atlasdet}.
Text snippets can be found on the web pages~\cite{pubcom-snippet}.
Comments and/or suggestions on improvements are very welcome and
should be given to the Publication Committee.

The \texttt{atlasphysics.sty} style file contains a set of particles,
reactions, etc.\ that may be useful. For example, the missing-$E_{\text{T}}$
symbol (\met{}) is defined there. 
A separate document~\cite{atlas-physics}
contains a list of the commands in \texttt{atlasphysics.sty}.


%-------------------------------------------------------------------------------
\section{Theory}
\label{sec:theory}
%-------------------------------------------------------------------------------

Sometimes combined with the introduction in shorter papers. 
This should be a brief overview of the theory relevant for this paper, with proper referencing.
This is a good place to define important quantities which will be used later.
State how the measurement of interest is actually performed, and how the parameters are extracted from the data.
If quantities are extracted with a fit to the data, describe the fit procedure;
if a limit is set, state what kind of limit-setting technique is used.
If these explanations need to be a bit longer, separate sections may be appropriate.

Give here a brief explanation of the theory relevant for this paper,
with proper referencing. Use, for example,
\verb|\cite{Glashow:1961tr,Weinberg:1967tq,sm_salam}| to cite the
Standard Model~\cite{Glashow:1961tr,Weinberg:1967tq,sm_salam}.

Define important quantities that will be used later.
State how the measurement of interest is actually performed, and how parameters are
extracted from the data.
If quantities are extracted with a fit to the data, describe the fit procedure.
If the explanation needs to be a bit longer, a separate section may be appropriate.


%-------------------------------------------------------------------------------
\section{Experimental setup}
\label{sec:atlas}
%-------------------------------------------------------------------------------

If necessary for your analysis, describe a subdetector in more detail.

The section on the experimental setup should also describe the dataset:
when were the data taken, under what special circumstances (if any),
and what was the integrated luminosity.

Examples of descriptions can be found in Appendix~\ref{app:atlasdet}.


%-------------------------------------------------------------------------------
\section{Monte Carlo}
\label{sec:MC}
%-------------------------------------------------------------------------------

All Monte Carlo generators and calculators used must be clearly named with the full version number and reference.
A brief description may be useful if it is a less well known generator.
It is usually necessary to say to what order it calculates (LO, NLO, NNLO...) 
and to specify any corrections, tunings or special parameter settings.
The PDF sets used should be listed, also with the full version numbers and references.
Statistics should be given if the number of simulated events gives rise to a non-negligible uncertainty.


%-------------------------------------------------------------------------------
\section{Event selection}
\label{sec:selection}
%-------------------------------------------------------------------------------

List the trigger and offline selection criteria, and number of events passing.


%-------------------------------------------------------------------------------
\section{Data characteristics}
\label{sec:control}
%-------------------------------------------------------------------------------

Discuss here the characteristics of the data, in terms of primary quantities and backgrounds.


%-------------------------------------------------------------------------------
\section{Systematic uncertainties}
\label{sec:syst}
%-------------------------------------------------------------------------------

Give a detailed list of systematic uncertainties, the method by which they were obtained,
and a justification of the resulting values.
Use ``systematic uncertainty'' instead of ``systematic error'.
The latter sounds as if you have made a mistake systematically.


%-------------------------------------------------------------------------------
\section{Results}
\label{sec:result}
%-------------------------------------------------------------------------------

State the results without interpretations here.


%-------------------------------------------------------------------------------
\section{Discussion}
\label{sec:discussion}
%-------------------------------------------------------------------------------

Put the results into the context of the theory or a model.
If the results lead to exclusion plots, make sure that it is clear which region on the plot is excluded.
Please also make sure that statistical interpretations and limit setting are done 
following the current ATLAS Statistics Tools and Guidelines

%-------------------------------------------------------------------------------
\section{Summary and conclusion}
\label{sec:summary}
%-------------------------------------------------------------------------------


Please make sure that there is a conclusion!
The concluding statements should reflect what is in the abstract.
Reiterate the main points of the paper and the primary results and conclusions.

Note that many readers look mostly at the title, abstract and conclusion.
The conclusion should be interesting enough to make them want to read the whole paper.
It is not good style just to repeat the abstract.
If your paper is short and only has one result quoted at the end of the paper,
then you should consider whether conclusions are necessary.

Try not to end your conclusions with a sentence such as
\enquote{All the results in this paper are in good agreement with the Standard Model,
the current world average and recent measurements by other experiments.}
This might lead a referee (internal or external) to wonder why it is worth publishing this paper!

Note that all figures and tables should be output before the start of the
\enquote{Summary and conclusion}.
This can be achieved using the macro \Macro{FloatBarrier} that is part of the 
\Package{placeins} package.


%-------------------------------------------------------------------------------
\section*{Acknowledgements}
\label{sec:acknowldge}
%-------------------------------------------------------------------------------

A standard template for the acknowledgements is available on the
web pages of the Publication Committee.
See reference~\cite{publication-policy} for the URL. 

The acknowledgements should be in an unnumbered section.


%-------------------------------------------------------------------------------
\clearpage
\appendix
\part*{Appendices}
\addcontentsline{toc}{part}{Appendices}
%-------------------------------------------------------------------------------

In a paper, an appendix is used for technical details that would otherwise disturb the flow of the paper.
Such an appendix should be printed before the Bibliography.
In an ATLAS paper, auxiliary plots and tables that are supposed to be made public 
should be collected in an appendix that has the title \enquote{Auxiliary material}.
This appendix should be printed after the Bibliography.

In an ATLAS note, use the appendices to include all the technical details of your work
that are relevant for the ATLAS Collaboration only (e.g.\ dataset details, software release used).

Use the following commands to start the Appendices section:
\begin{verbatim}
   \clearpage
   \appendix
   \part*{Appendices}
   \addcontentsline{toc}{part}{Appendices}
\end{verbatim}

In this document the appendices contain some general remarks on the style to use in ATLAS papers.
More technical details on how to use the ATLAS \LaTeX\ templates to help you follow 
these recommendations can be found in other documents:
\begin{itemize}
\item ATLAS style guide~\cite{atlas-style}.
\item Users guide to the ATLAS \LaTeX\ package~\cite{atlas-latex};
\item Guide to references and \BibTeX\ in ATLAS documents~\cite{atlas-bibtex};
\item ATLAS physics symbols~\cite{atlas-physics};
\item Guide to formatting tables for ATLAS documents~\cite{atlas-tables};
\item Rounding – ATLAS Recommendations~\cite{atlas-rounding};
\item Report of the ATLAS figures task force~\cite{atlas-figures}.
\end{itemize}

ATLAS papers should use the \Option{atlasstyle} option with the class \Package{atlasdoc}.
The default fonts for ATLAS documents are \Package{newtxtext} and \Package{newtxmath}.
These are more modern variants of the package \Package{txfonts} that has been used in
ATLAS documents in the past.
However, arXiv currently uses an older version of \TeX\ Live (2011) that does not include the
\Package{newtx} fonts. Hence, in ATLAS papers, please add the option \Option{txfonts} to \Package{atlasdoc}.

%-------------------------------------------------------------------------------
\section{General style}
\label{sec:english}
%-------------------------------------------------------------------------------

We recommend the use of British English. However, whatever you decide
to choose, be consistent throughout the paper. For much more detailed
information on writing, spelling and typographic style, etc. please
see the ATLAS Style Guide~\cite{atlas-style}. 
The ATLAS Glossary~\cite{atlas-glossary} contains a list of ATLAS detector acronyms.

Use italics for emphasis sparingly: too many italicised words defeat
their purpose. When you do italicise a word, really italicise it: do
not use math mode! Note the difference between \emph{per se}
(\verb|\emph{per se}|) and $per se$ (\verb+$per se$+). Abbreviations
like i.e., e.g., etc., and et al.\ should \emph{not} be italicised!
For program names we recommend to use small capitals:
\verb|\textsc{Pythia}}| produces \textsc{Pythia}.


%-------------------------------------------------------------------------------
\section{Tables}
\label{sec:table}
%-------------------------------------------------------------------------------

All tables appearing in the paper must be mentioned in the text. The
tables should appear in the same order as mentioned in the text. Do
not abbreviate the word \enquote{Table}. Capitalise the first word of any
row or column heading. Label all rows/columns, including units if
appropriate. Units are given in brackets, e.g.\ \enquote{Energy [\si{\GeV}]}.
Horizontal and vertical lines should be kept to a minimum and only used 
as necessary to make the table entries clear.
The table caption appears below the table.
Any symbols/variables used in the table must be defined in the
caption.  Always use labels to refer to tables and figures. Note that
the label must be included in the body of the caption.

If table entries include uncertainties, the entries should be aligned
on the \enquote{$\pm$} sign. Numbers should be aligned on the decimal
point. Otherwise, column entries should be left-aligned.

More details on the style to be followed can be found in Section~3.4 of the \enquote{ATLAS Style Guide}~\cite{atlas-style}.
Example tables and hints on how to obtain the desired format can be found in a separate document~\cite{atlas-tables}.
The note on rounding~\cite{atlas-rounding} contains more examples on how to format tables.


%-------------------------------------------------------------------------------
\section{Figures}
\label{sec:figure}
%-------------------------------------------------------------------------------

All figures appearing in the paper must be mentioned in the text.
The figures should appear in the same order as mentioned in the text.
At the beginning of a sentence, use the full word \enquote{Figure}.
Within a sentence, the abbreviation \enquote{Fig.} may be used.
If a figure appears in two or more parts, refer to it as
\enquote{Fig.~1(a)} and \enquote{Fig.~1(b)}. Both \enquote{(a)} and \enquote{(b)} should
appear in the text, in the figure, and in the caption.
The word \enquote{ATLAS} (or \enquote{ATLAS Preliminary}, if appropriate) should
appear prominently somewhere in the figure. This becomes important when
the figure is copied and shown out of context. If appropriate, it
is useful to include information about the luminosity corresponding
to a figure.

All axes must be labelled, including units (i.e.\ \enquote{Energy [\si{\GeV}]}).
The vertical axis units should specify the bin width, unless
arbitrarily normalized. A legend box explaining all plotting symbols
must appear somewhere in the figure.

The caption should be placed below the figure.  All lines, all
plotting symbols, and all variables used in the figure must be defined
in the caption. Do not refer to any characteristic that is not
distinguishable in black-and-white.
If relevant, the normalisation method of the plot should be specified.

More details on the style to be followed can be found in Section~3.3 of the \enquote{ATLAS Style Guide}~\cite{atlas-style}.
The Publication Committee has a root macro to create figures in
ATLAS style, it can be found on the Publication Committee web pages~\cite{atlas-rootstyle}.
Use this style consistently throughout the paper.


%-------------------------------------------------------------------------------
\section{References}
\label{sec:refs}
%-------------------------------------------------------------------------------

Only cite permanent, publicly available, or ATLAS approved references.
Private references, not available to the general public, should be
avoided. Caution should be used when referring to ATLAS notes.
Only reference approved notes. Do not reference COM or INT notes,
as these are not available outside ATLAS.

Whenever possible, cite the article's journal rather than its preprint number. 
If possible, the arXiv number should be given in addition.
Always double check references when copying them from another source.

Referencing styles are journal-dependent.
This is not a problem if the ATLAS preprint style is used,
as the journals will adapt the document to their style.

For further information on \BibTeX\ and on the standard ATLAS style for referencing, 
see the \enquote{ATLAS \BibTeX\ Guide}~\cite{atlas-bibtex}.
More details on what to pay attention to can be found in Section~3.6 of the \enquote{ATLAS Style Guide}~\cite{atlas-style}.


%-------------------------------------------------------------------------------
\section{Remarks on units and symbols}
\label{sec:units}
%-------------------------------------------------------------------------------

It is highly recommended to use a units package to format your units properly.
The package \texttt{siunitx} works very well and is the package of choice.
Alternatives include \texttt{units} and \texttt{hepunits},
which is based on \texttt{SIunits}.

If you use a units package, it automatically set SI units in an upright font
and leaves a small space between
the value and the units (e.g.\ \SI{12}{\mm}). 
In addition, it makes sure the value and unit end up on the same line: \verb|\SI{12}{\mm}|.
Natural units, where $c=\hbar=1$, should be used for all
ATLAS publications.
\ifthenelse{\ATTeXLiveVersion < 2011}{%
Masses are therefore in \si{\GeV}, not \si[per=slash]{\GeV\per\clight}.
}{%
Masses are therefore in \si{\GeV}, not \si[per-mode=symbol]{\GeV\per\clight}.
}
Use math mode for all symbols (e.g. use $c$ (\verb|$c$|) rather than simply c). 
Momentum should be in lower case \verb+$p$+. 
Transverse momentum is a lower case $p$ with an upper case roman $\text{T}$ subscript: 
\verb|\pT| produces \pT.
Energy is an upper case \verb+$E$+, \verb+\ET+ produces \ET.  
Use \verb|\mathcal| mode for luminosity $\mathcal{L}$ or aplanarity
$\mathcal{A}$.
%Use \verb|\mathscr| mode for luminosity $\mathscr{L}$ $\mathcal{L}$ or aplanarity
%$\mathscr{A}$ $\mathcal{A}$, including the package \verb|mathrsfs.sty|.

Trigonometric functions should be in roman type. Natural logarithm
should be ln and log base 10 is log.  When in math mode, use
\verb+$\ln$, $\sin$,+ etc. We recommend to specify the base of the
logarithm: \verb+$\log_{10}$+.

If your note makes use of cones, for example cone-jets, explain that
these cones are constructed in $\eta$-$\phi$ space, and define $\eta$.

Add the word \emph{events} as the unit when quoting the number of
events: \enquote{The resulting background is $4.0 \pm 1.3$ events.}.  The
number of expected events should be written as $N_{\text{pred}}$ rather
than $N_{\text{exp}}$, since the latter could also mean experimental.

For particle names and symbols, ATLAS uses the standards of the
Particle Data Book. Intermediate vector bosons should be called
\emph{W boson(s)} and \emph{Z boson(s)}, not just \emph{W's} or
\emph{Ws}. The Z boson should not have a superscript of 0. W without
the word boson attached may be used in \emph{W pair production}, and
similar phrases.  Other particle names should be spelled out when used
in a sentence: muon(s), electron(s), tau lepton(s). \emph{Top quark}
should be used instead of \emph{top} in most places: 
say \enquote{top-quark mass} instead of \enquote{top mass}.
Top quark and bottom quark may be shortened to $t$ quark and $b$ quark. 
The neutrino symbol $\nu$ should not have any subscripts, unless necessary for
understanding.
For the \Jpsi{} use the command \verb+\Jpsi+ from 
\texttt{atlasphysics.sty}: it will produce a lower case $\psi$.

When in doubt, use the PDG style.


%-------------------------------------------------------------------------------
\section{The \texttt{atlasdoc} class}
\label{sec:atlasdoc}
%-------------------------------------------------------------------------------

This paper has been typeset using the \texttt{atlasdoc.cls} class.
This class implements the ATLAS template and can be used for papers, preprints,
notes. The class is available from the web pages of the
Publication Committee and from SVN.
It also contains these guidelines and related files.

Details on how to use the templates can be found in a separate document
\texttt{atlas\_latex.pdf}~\cite{atlas-latex}.

The style file \File{atlaspackage.sty} loads packages that are usually used in documents.
Several options exist to turn on or off the loading of various packages.

The style file \texttt{atlasphysics.sty} defines a lot of useful
macros for particles and more. See the separate document
\texttt{atlas\_physics.pdf}~\cite{atlas-physics} for details.


%-------------------------------------------------------------------------------
% Bibliography
%-------------------------------------------------------------------------------
\printbibliography
% This document uses biblatex and biber to process the bibliography.
% If you want to use BibTeX you need to use the syntax below.
% \bibliographystyle{../atlasBibStyleWoTitle}
% \bibliography{atlas_paper,../../bibtex/bib/ATLAS,../atlas_latex.bib}


%-------------------------------------------------------------------------------
\clearpage
\appendix
\part*{Auxiliary material}
\addcontentsline{toc}{part}{Auxiliary material}
%-------------------------------------------------------------------------------


%-------------------------------------------------------------------------------
\section{ATLAS detector descriptions}
\label{app:atlasdet}
%-------------------------------------------------------------------------------

These descriptions can be found in the file \File{templates/atlas-detector.tex}.

% Footnote with ATLAS coordinate system
\newcommand{\AtlasCoordFootnote}{%
ATLAS uses a right-handed coordinate system with its origin at the nominal interaction point (IP)
in the centre of the detector and the $z$-axis along the beam pipe.
The $x$-axis points from the IP to the centre of the LHC ring,
and the $y$-axis points upwards.
Cylindrical coordinates $(r,\phi)$ are used in the transverse plane, 
$\phi$ being the azimuthal angle around the $z$-axis.
The pseudorapidity is defined in terms of the polar angle $\theta$ as $\eta = -\ln \tan(\theta/2)$.
Angular distance is measured in units of $\Delta R \equiv \sqrt{(\Delta\eta)^{2} + (\Delta\phi)^{2}}$.}

%-------------------------------------------------------------------------------
\subsection{ATLAS detector example from a Letter}
\label{sec:atlas1}
%-------------------------------------------------------------------------------

The ATLAS experiment~\cite{PERF-2007-01} at the LHC is a multi-purpose particle detector
with a forward-backward symmetric cylindrical geometry and a near $4\pi$ coverage in 
solid angle.\footnote{\AtlasCoordFootnote}
It consists of an inner tracking detector surrounded by a thin superconducting solenoid
providing a \SI{2}{\tesla} axial magnetic field, electromagnetic and hadron calorimeters, and a muon spectrometer.
The inner tracking detector covers the pseudorapidity range $|\eta| < 2.5$.
It consists of silicon pixel, silicon micro-strip, and transition radiation tracking detectors.
Lead/liquid-argon (LAr) sampling calorimeters provide electromagnetic (EM) energy measurements
with high granularity.
A hadron (iron/scintillator-tile) calorimeter covers the central pseudorapidity range ($|\eta| < 1.7$).
The end-cap and forward regions are instrumented with LAr calorimeters
for both EM and hadronic energy measurements up to $|\eta| = 4.9$.
The muon spectrometer surrounds the calorimeters and is based on
three large air-core toroid superconducting magnets with eight coils each.
Its bending power is in the range from \num{2.0} to \SI{7.5}{\tesla\metre}.
It includes a system of precision tracking chambers and fast detectors for triggering.
A three-level trigger system is used to select events.
The first-level trigger is implemented in hardware and uses a subset of the detector information
to reduce the accepted rate to at most \SI{75}{\kilo\hertz}.
This is followed by two software-based trigger levels that
together reduce the accepted event rate to \SI{400}{\hertz} on average
depending on the data-taking conditions during 2012.


%-------------------------------------------------------------------------------
\subsection{ATLAS detector example from a paper making use of the whole detector}
\label{sec:atlas2}
%-------------------------------------------------------------------------------

The ATLAS detector~\cite{PERF-2007-01} at the LHC covers nearly the entire solid angle around the collision point.
It consists of an inner tracking detector surrounded by a thin superconducting solenoid, electromagnetic and hadronic calorimeters,
and a muon spectrometer incorporating three large superconducting toroid magnets.
The inner-detector system (ID) is immersed in a \SI{2}{\tesla} axial magnetic field 
and provides charged particle tracking in the range $|\eta| < 2.5$.

The high-granularity silicon pixel detector covers the vertex region and typically provides three measurements per track, 
the first hit being normally in the innermost layer.
It is followed by the silicon microstrip tracker which usually provides four two-dimensional measurement points per track.
These silicon detectors are complemented by the transition radiation tracker,
which enables radially extended track reconstruction up to $|\eta| = 2.0$. 
The transition radiation tracker also provides electron identification information 
based on the fraction of hits (typically 30 in total) above a higher energy deposit threshold corresponding to transition radiation.

The calorimeter system covers the pseudorapidity range $|\eta| < 4.9$.
Within the region $|\eta|< 3.2$, electromagnetic calorimetry is provided by barrel and 
endcap high-granularity lead/liquid-argon (LAr) electromagnetic calorimeters,
with an additional thin LAr presampler covering $|\eta| < 1.8$,
to correct for energy loss in material upstream of the calorimeters.
Hadronic calorimetry is provided by the steel/scintillating-tile calorimeter,
segmented into three barrel structures within $|\eta| < 1.7$, and two copper/LAr hadronic endcap calorimeters.
The solid angle coverage is completed with forward copper/LAr and tungsten/LAr calorimeter modules
optimised for electromagnetic and hadronic measurements respectively.

The muon spectrometer (MS) comprises separate trigger and
high-precision tracking chambers measuring the deflection of muons in a magnetic field generated by superconducting air-core toroids.
The precision chamber system covers the region $|\eta| < 2.7$ with three layers of monitored drift tubes,
complemented by cathode strip chambers in the forward region, where the background is highest.
The muon trigger system covers the range $|\eta| < 2.4$ with resistive plate chambers in the barrel, and thin gap chambers in the endcap regions.
A three-level trigger system is used to select interesting events~\cite{PERF-2011-02}.
The Level-1 trigger is implemented in hardware and uses a subset of detector information
to reduce the event rate to a design value of at most \SI{75}{\kHz}.
This is followed by two software-based trigger levels which together reduce the event rate to about \SI{200}{\Hz}.




\end{document}
